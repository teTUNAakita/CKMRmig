\documentclass[AMA,STIX1COL]{WileyNJD-v2}
\usepackage{moreverb}
\usepackage{lineno}
\linenumbers
\usepackage[doublespacing]{setspace}

\usepackage{amsfonts}
\usepackage{amssymb}
\usepackage{amstext}
\usepackage{amsmath}

\newcommand\BibTeX{{\rmfamily B\kern-.05em \textsc{i\kern-.025em b}\kern-.08em
T\kern-.1667em\lower.7ex\hbox{E}\kern-.125emX}}

\articletype{Research article, \today}%

\received{<day> <Month>, <year>}
\revised{<day> <Month>, <year>}
\accepted{<day> <Month>, <year>}

%\raggedbottom

\begin{document}

\title{Estimating migrant rates using kinship pairs \protect\thanks{Running head line: Estimating migrant rates using kinship pairs}}

\author[1]{Tetsuya Akita}
\authormark{Akita}


\address[1]{\orgdiv{Fisheries Data Sciences Division}, \orgname{Fisheries Resources Institute}, \orgaddress{\state{Kanagawa}, \country{Japan}}}

\corres{*Tetsuya Akita, Fisheries Data Sciences Division, Fisheries Resources Institute, FRA, 236-8648 Kanagawa, Japan. \email{akitatetsuya1981@affrc.go.jp}}

%\presentaddress{Present address}

\abstract[Abstract]{The Abstract must not exceed 350 words and should list the main results and conclusions, using simple, factual, numbered statements:

Point 1: set the context for and purpose of the work;

Point 2: indicate the approach and methods;

Point 3: outline the main results;

Point 4: identify the conclusions and the wider implications.

1. aaa\\
2. bbb

本ノートでは,多回繁殖(iteroparity)する種を対象として,半兄弟ペア数から親魚による2つのサブ集団間の移動率を推定する理論($\hat{M}=N_1N_2H_{\rm obs}/(4n_1n_2)$)を紹介する。この推定式は、親をサンプルする通常の標識放流と同じである。もし、繁殖ポテンシャルの分布が集団間で異なっていても上記の式で問題ないことを示した。一方,サンプル数としては$\sqrt {N}$以上のオーダーが求められる。}

\keywords{Close-kin mark recapture; Half-sibling; Population structure}

\maketitle

%\footnotetext{\textbf{Abbreviations:} ANA, anti-nuclear antibodies; APC, antigen-presenting cells; IRF, interferon regulatory factor}


\section{Introduction}\label{sec1}

Estimating the number of migrants among the populations is essential for understanding population structure and dynamics, which is a fundamental process for implementing conservation and wildlife management. Especially in fisheries management, migration rate is a critical component for delineating a synchrony among the population, providing appropriate management units. There are essentially two approaches for estimating the migrant number. The first is to use the mark-recapture method estimate it directly from the population dynamics modeling with survey data. The second is to ...

移動率の推定は,対象生物の理解に重要なだけでなく,外来種管理・希少種保全・資源の持続的利用にとっても重要である。 特に、水産資源管理にとっては、管理単位を決めるための指標として、移動率はクリティカルな指標の一つである。既存の方法論としては,大きく2種類に大別される。1つ目は標識再捕法であり,2つ目は集団遺伝学的な方法である。前者については,サンプル対象を再捕獲が見込める親個体に限定せざるを得ない,非侵襲的な方法でのラベリングが困難であること,推定精度に見合うサンプルサイズが現実的でないこと,報告率が大きくばらつく等が問題となっている。後者については,近年(すなわち$N_{\rm e}$よりも短い世代数)の集団構造については解像度がないこと,移動個体数が大きいと遺伝的には同一の集団とみなされること等が問題となっている。

本ノートでは,近親標識法(Close-kin mark-recapture, CKMR)という近年発達した方法論を用いて,移動率の推定を試みる。CKMRとは,親集団が小さいほど近親ペアが見つかりやすいことに着目した方法である。今回は,多回繁殖する生物を対象として,2つのサブ集団間で見つかる0歳の半兄弟(Half-sibling, HS)ペアを用いて,そのペアの親候補となる個体の数(もしくは移動率)を推定する。ただし,移動前の集団および移動後の集団における親魚尾数が,移動個体数の推定には必要となる。

応用例として,3年間にわたって2つのサブ集団からサンプルされた当歳魚から見出される半兄弟ペアをもとに,個体数・移動率・生存率を推定する枠組みを提示する。そこでは,同じサブ集団から年を跨いで見つかる半兄弟ペアに基づく関係式と,今回新たに開発した関係式を組み合わせて用いる。

\section{Theory}\label{sec2}
The main symbols used in the current paper are summarized in Table \ref{symbols}.

\renewcommand{\arraystretch}{0.6}
\begin{table}[tb]
\begin{center}
   \caption[]{The list of mathematical symbols employed in the main text}
    \textbf {}\\[-4mm]
    \begin{tabular}{llc} \hline
       & & \\
	$n_{\rm P,1}, n_{\rm P, 2}$			& Sample number of parents in the subpopulation 1 and 2\\ 
		                						& \\
	$n_{\rm O, 1}, n_{\rm O, 2}$			& Sample number of offspring in the subpopulation 1 and 2\\ 
		                						& \\
	$N_{1}, N_{2}$						& Number of parents in the subpopulation 1 and 2 when sampled offspring are born\\
		                						& \\
	$M$								& Number of migrants of parents from the subpopulation 1 to the subpopulation 2\\
		                						& \\
	$\pi_{\rm PO,1}, \pi_{\rm PO,2}$		& Probability that a randomly selected pair (mother and offspring) \\
	                							& shares a parent-offspring relationship within the subpopulation 1 and 2\\
									& \\
	$\pi_{\rm PO, bet}$					& Probability that a randomly selected pair (parent and offspring in the population 1 and 2, respectively) \\
	                							& shares a parent-offspring relationship between the subpopulation 1 and 2\\
									& \\
	$\pi_{\rm HS, bet}$					& Probability that a randomly selected pair (two offspring in the population 1 and 2) \\
	                							& shares a half-sibling relationship between the subpopulation 1 and 2\\
					                			& \\
	$\lambda_{i,1}, \lambda_{j,2}$			& Expected number of surviving offspring of parent $i$ and $j$ at sampling in the subpopulation 1 and 2\\
		                						& \\
	$\lambda_{i, M}$					& Expected number of surviving offspring of migrant $i$ at sampling\\
		                						& \\
	$k_{i,1}, k_{j,2}$					& Number of surviving offspring born to parent $i$ and $j$ in the subpopulation 1 and 2\\ 
	                							& \\
	$H_{\rm PO, 1}, H_{\rm PO, 2}$		& Number of parent-offspring pairs observed in samples within the subpopulation 1 and 2\\ 
	                							& \\
	$H_{\rm PO, bet}$					& Number of parent-offspring pairs observed in samples between the subpopulation 1 and 2\\ 
	                							& \\
	$H_{\rm HS, bet}$					& Number of half-sibling pairs observed in samples between the subpopulation 1 and 2\\ 
	                							& \\              		
	\hline
    \end{tabular}
    \label{symbols} 
\end{center} 
\end{table}
\renewcommand{\arraystretch}{1}

\subsection{Assumptions}

この章では,全てに共通するモデリングの仮定を述べた後,シンプルなモデルから徐々に複雑なモデルへと移行する。具体的には,最初にシンプルなモデルについて記述して,
\begin{itemize}
  \item 移動が双方向
  \item 性比1:1およびランダム交配を仮定して両性を考慮
  \item 繁殖ポテンシャルが一定でない
  \item 移動しやすさが繁殖ポテンシャルに依存
\end{itemize}
といった要素を追加していく。また,どのモデルでも共通して仮定しているのは以下の5点である。
\begin{enumerate}
  \item 年をまたぐ多回繁殖
  \item 親魚によるサブ集団間の移動は,繁殖シーズンの間に起きる
  \item サンプルされる当歳魚(0歳)は,必ずサンプルされたサブ集団で産まれる
  \item 当歳魚かどうかの判断は100%可能
  \item サンプル間の近親判別では,100%の成功率で半兄弟ペアか否かがなされる
\end{enumerate}
また,性成熟した個体を親と呼ぶことにする。

%%%FIGURE 1%%%
\begin{figure}[!h]
	\begin{center}
		\includegraphics[width=0.3\textwidth]{/Users/akita/Dropbox/research/close_kin/Mig/Figs/cartoon.eps}
		\caption{Example of relationships between mothers and their offspring number. The open, gray, and filled circles represent mothers, their eggs, and their offspring, respectively. The area of an open circle indicates the degree of reproductive potential of each mother (i.e., $\lambda_i$). The dotted and thin arrows denote mother--egg and egg--offspring relationships, respectively. The symbol x denotes a failure to survive at sampling. Sampled individuals are denoted with squares. In this example, $M=6$, }
		\label{cartoon}
	\end{center}
\end{figure}
%%%%%%%%%%%

\subsection{HSP-based model}

\begin{align}
\pi_{\rm HS,bet} &= \frac{2M}{N_{1}} \frac{2M}{N_{2}} \frac{1}{M} \nonumber\\
&= \frac{4M}{N_{1}N_{2}}
\label{HS-1}
\end{align}

\begin{align}
\mathbb{E}[H_{\rm HS, bet}] &= \pi_{\rm HS,bet} n_{\rm O, 1} n_{\rm O, 2} \nonumber\\
&= \frac{ 4 n_{\rm O, 1} n_{\rm O, 2} M } { N_{1} N_{2} } \\
&= \frac{ 4 n_{\rm O, 1} n_{\rm O, 2} m } { N_{2} }
\label{HS-2}
\end{align}

\begin{align}
\widehat{M_1} &= \frac{N_{1} N_{2} \widetilde{H}_{\rm HS, bet} } {4 n_{\rm O, 1} n_{\rm O, 2} }
\label{HS-3}
\end{align}

\begin{align}
\widehat{m_1} &= \frac{N_{2} \widetilde{H}_{\rm HS, bet} } {4 n_{\rm O, 1} n_{\rm O, 2} }
\label{HS-3}
\end{align}

より一般的な説明は付録に載せた。

\subsection{POP-based model}

\begin{align}
\pi_{\rm PO,bet} &= \frac{2M}{N_{1}} \frac{M}{N_{2}} \frac{1}{M} \nonumber\\
&= \frac{2M}{N_{1}N_{2}}
\label{PO-1}
\end{align}

\begin{align}
\mathbb{E}[H_{\rm PO, bet}] &= \pi_{\rm PO,bet} n_{\rm O, 1} n_{\rm P, 2} \nonumber\\
&= \frac{ 2 n_{\rm O, 1} n_{\rm P, 2} M } { N_{1} N_{2} }
\label{PO-2}
\end{align}

\begin{align}
\widehat{M_2} &= \frac{N_{1} N_{2} \widetilde{H}_{\rm PO,bet} } {2 n_{\rm O, 1} n_{\rm P, 2} }
\label{PO-3}
\end{align}

\begin{align}
\widehat{m_2} &= \frac{N_{2} \widetilde{H}_{\rm PO,bet} } {2 n_{\rm O, 1} n_{\rm P, 2} }
\label{PO-4}
\end{align}

\subsection{Required sample size}

\begin{align}
n_{\rm O, 1} n_{\rm O, 2} &> \frac{ N_{2} } { 4m }
\label{RS-1}
\end{align}

\begin{align}
n_{\rm O, 1} n_{\rm P, 2} &> \frac{ N_{2} } { 2m }
\label{RS-2}
\end{align}

\section{Applications}\label{sec3}

\subsection{Efficient use of kinship pairs found between subpopulations}

\begin{align}
\widehat{M_3} &= \frac{ N_{1} N_{2} \left(\widetilde{H}_{\rm HS,bet} + \widetilde{H}_{\rm PO,bet}\right) } {2n_{\rm O, 1} \left(2 n_{\rm O, 2} + n_{\rm P, 2}\right) }
\label{AP-1}
\end{align}

\begin{align}
\widehat{m_3} &= \frac{ N_{2} \left(\widetilde{H}_{\rm HS,bet} + \widetilde{H}_{\rm PO,bet}\right) } {2n_{\rm O, 1} \left(2 n_{\rm O, 2} + n_{\rm P, 2}\right) }
\label{AP-2}
\end{align}

サンプル数の少なさを補えるかも。

\subsection{Parent number can be estimated by PO pairs found within a subpopulation}

\begin{align}
\widehat{N_2} &= \frac{ 2 n_{\rm O, 2} n_{\rm P, 2} + 1 }{\widetilde{H}_{\rm PO, 2} + 1} 
\label{AP-3}
\end{align}
subpop2の親子もわかるはず。$N_2$を推定値で置き換えて

\begin{align}
\widehat{M_4} &=  \frac{ N_{1} \left(\widetilde{H}_{\rm HS,bet} + \widetilde{H}_{\rm PO,bet}\right) \left(2 n_{\rm O, 2} n_{\rm P, 2} + 1\right)} {2n_{\rm O, 1} \left(2 n_{\rm O, 2} + n_{\rm P, 2}\right) \left(\widetilde{H}_{\rm PO, 2} + 1\right)}
\label{AP-4}
\end{align}

migration rateを推定すると考えると、

\begin{align}
\widehat{m_4} &=  \frac{ \left(\widetilde{H}_{\rm HS,bet} + \widetilde{H}_{\rm PO,bet}\right) \left(2 n_{\rm O, 2} n_{\rm P, 2} + 1\right)} {2n_{\rm O, 1} \left(2 n_{\rm O, 2} + n_{\rm P, 2}\right) \left(\widetilde{H}_{\rm PO, 2} + 1\right)}
\label{AP-5}
\end{align}

同様に、$N_1$も推定値で置き換えて
\begin{align}
\widehat{N_1} &= \frac{ 2 n_{\rm O, 1} n_{\rm P, 1} + 1 }{\widetilde{H}_{\rm PO, 1} + 1} 
\label{AP-6}
\end{align}

\begin{align}
\widehat{M_5} &=  \frac{ \left(\widetilde{H}_{\rm HS,bet} + \widetilde{H}_{\rm PO,bet}\right) \left(2 n_{\rm O, 1} n_{\rm P, 1} + 1\right) \left(2 n_{\rm O, 2} n_{\rm P, 2} + 1\right)} {2n_{\rm O, 1} \left(2 n_{\rm O, 2} + n_{\rm P, 2}\right) \left(\widetilde{H}_{\rm PO, 1} + 1\right) \left(\widetilde{H}_{\rm PO, 2} + 1\right)}
\label{AP-7}
\end{align}

\subsection{Individual-based model}

\renewcommand{\arraystretch}{0.6}
\begin{table}[tb]
\begin{center}
   \caption[]{Summary of proposed estimators for required parameters and kinship types}
    \textbf {}\\[-0mm]
    \begin{tabular}{ccccccccc} \hline
       	\\
	Estimator			& $N_1$ 		& $N_2$ 		& $n_{\rm O, 1}$	& $n_{\rm O, 2}$	& $n_{\rm P, 1}$	& $n_{\rm P, 2}$	& Required kinship type\\
	\\
	\hline
	\\
	$\widehat{M_1}$	& given 		& given 		& \checkmark		& \checkmark 		&				&				& HSP\\ 
		                						& \\
	$\widehat{M_2}$	& given		& given		& \checkmark		& 				&				& \checkmark 		& POP\\
		                						& \\
	$\widehat{M_3}$	& given		& given		& \checkmark		&  \checkmark		&				& \checkmark 		& HSP \& POP\\
		                						& \\
	$\widehat{M_4}$	& given		& estimated	& \checkmark		&  \checkmark		&				& \checkmark 		& HSP \& POP\\
		                						& \\
	$\widehat{M_5}$	& estimated	& estimated	& \checkmark		&  \checkmark		& \checkmark		& \checkmark 		& HSP \& POP\\
									& \\
	$\widehat{m_1}$	& ---			& given		& \checkmark		& \checkmark 		&				&				& HSP\\ 
									& \\
	$\widehat{m_2}$	& ---			& given		& \checkmark		& 				&				& \checkmark 		& POP\\					                			& \\
	$\widehat{m_3}$	& ---			& given 		& \checkmark		&  \checkmark		&				& \checkmark 		& HSP \& POP\\
		                						& \\
	$\widehat{m_4}$	& ---			& estimated	& \checkmark		&  \checkmark		& 				& \checkmark 		& HSP \& POP\\
		                						& \\
	\hline
    \end{tabular}
    \label{parameter} 
\end{center} 
\end{table}
\renewcommand{\arraystretch}{1}

\begin{comment}%%%%%%%%%%%%%%%
まず最初に,最もシンプルなモデルを考える。$t$年にサブ集団1にいる$N_{1,t}$個体の母親を考えよう。$t$年に子供を産んだ後,$N_{1,t}$個体のうち$M_{1\to2}$個体がサブ集団2に移動し,$t+1$年にサブ集団2にいる母個体$N_{2,t+1}$の一部として子供を産むとする。この時,サブ集団1から$t$年に産まれた子供とサブ集団2から$t+1$年に産まれた子供が,母を共有する兄弟(Maternal half Sibling, MHS)である確率$\pi_{\rm b,1\to2}$を考える。どの母親も同じような繁殖ポテンシャルを持つと仮定すると,移動する母親の子供をサブ集団1でサンプルする確率は$M_{1\to2}/N_{1,t}$となる。同様に,移動する母親の子供をサブ集団2においてサンプルする確率は$M_{1\to2}/N_{2,t+1}$である。それぞれでサンプルした子供の母親が同一である確率は$1/M_{1\to2}$であることより,サンプルした2個体がMHSペアとなる確率が以下のように得られる。
\begin{align}
\pi_{\rm b,1\to2} &= \frac{M_{1\to2}}{N_{1,t}} \frac{M_{1\to2}}{N_{2,t+1}} \frac{1}{M_{1\to2}} \nonumber\\
&= \frac{M_{1\to2}}{N_{1,t}N_{2,t+1}}
\label{1-1}
\end{align}
サブ集団1からサブ集団2に移動する割合$m_{1\to2}$を導入することで,上式は
\begin{align}
\pi_{\rm b,1\to2} &= \frac{m_{1\to2}}{N_{2,t+1}}
\label{1-2}
\end{align}
と書ける。$M_{1\to2} = m_{1\to2}N_{1,t}$とした。$m_{1\to2}$には移動だけでなく$t+1$年の産卵までの生存確率が含まれている点に注意。

MHSペアとなる確率が得られれば,サンプル数に応じて見つかるMHSペア$H_{1\to2}$の期待値が得られる。
\begin{align}
\mathbb{E}[H_{1\to2}] &= \pi_{\rm b,1\to2} n_{1,t} n_{2,t+1} \nonumber\\
&= \frac{m_{1\to2}n_{1,t} n_{2,t+1}} {N_{2,t+1}}
\label{1-3}
\end{align}
$n_{1,t}, n_{2,t+1}$は,それぞれサブ集団1($t$年)でのサンプル数,サブ集団2($t+1$年)でのサンプル数である。この結果から,平均的にみて少なくとも1ペア以上を見出すには,(親魚尾数)の平方根のオーダー以上のサンプル数が要求されることがわかる。以上より,見出されたMHSペア数から母親の移動率に関する線形推定量が得られる。
\begin{align}
\widehat{m_{1\to2}} &= \frac{N_{2,t+1} H_{1\to2,{\rm obs}} } {n_{1,t}n_{2,t+1}}
\label{1-4}
\end{align}
ハットは推定量を表し,obsの添字は観測値であることを表す。

\subsection{Two-way movement}

次に,反対方向の移動について考える。すなわち,$t$年にサブ集団2にいる$N_{2,t}$個体の母親から,$M_{1\to2}$個体が$t+1$年にサブ集団1へ移動すること考えよう。これまで同様に考えて,MHSペアとなる確率は,
\begin{align}
\pi_{\rm b,2\to1} &= \frac{m_{2\to1}}{N_{1,t+1}}
\label{2-1}
\end{align}
のようになり,移動率に関する推定量
\begin{align}
\widehat{m_{2\to1}} &= \frac{N_{1,t+1} H_{2\to1,{\rm obs}} } {n_{2,t}n_{1,t+1}}
\label{2-2}
\end{align}
が得られる。すなわち,移動が一方向であろうが双方向であろうが,各方向の移動率に関する推定量は式\ref{1-4}と式\ref{2-2}から独立に得られる。したがって,以降の導出ではサブ集団1からサブ集団2に移動する個体についてだけ考え,添字を省略する。例えば,$N_1$という記載は,$t$年に親魚として存在した個体数を表す。

\subsection{Two-sex}

ここまでは、母親を介した兄弟ペアと、母親の個体数およびその移動について考えてきた。しかしながら、母親を介した兄弟か父親を介した兄弟かを見分けるには、ミトコンドリアの情報を用いた類推が必須である点、当歳魚のサンプル数が同程度でも区別しない半兄弟ペアの半数程度しか発見されない点などから、実際の応用では区別しない半兄弟ペアを用いるほうが望ましい。そこで、まずは性比=1:1(すなわち,$N_{\rm male}=N_{\rm female}=N_1/2$かつ$M_{\rm male}=M_{\rm female}=M/2$)でランダム交配を仮定した集団を考えよう。この時、サブ集団1でサンプルした当歳魚の親が移動する親個体である確率は$M_{\rm male}/N_{\rm male} + M_{\rm female}/N_{\rm female}=2M/N_1$となる。同様に,移動した親の子供をサブ集団2においてサンプルする確率は$2M/N_2$であることから,
\begin{align}
\pi &= \frac{2M}{N_{1}} \frac{2M}{N_{2}} \frac{1}{M} \nonumber\\
&= \frac{4M}{N_{1}N_{2}}
\label{3-1}
\end{align}
が得られる。その結果,移動率の推定値は新たに係数4が分母に加わって
\begin{align}
\widehat{m} &= \frac{N_{2} H_{\rm obs} } {4n_{1}n_{2}}
\end{align}
となる。

ここでは応用例として,3年間にわたって2つのサブ集団からサンプルされた当歳魚から見出される半兄弟ペアをもとに,個体数($N$)・移動率($m$)・生存率($s$)を推定する理論的な枠組みを提示する。サブ集団1に生息する親魚数を、$N_1(t), N_1(t+1), N_1(t+2)$のように表記し、他のパラメータについても同様とする。個体数については6つのパラメータ、移動率および生存率は4つのパラメータとなり、合計14のパラメータが考えられる($t+2$年の移動率および生存率は含まれない点に注意)。まずは、これら14個のパラメータは独立に決まるとしよう(すなわち、$N_1(t)$と$N_1(t+1)$は無関係で、$s_1(t)$と$s_1(t+1)$は独立に決める、など)。なお、この応用例では、移動する親魚は繁殖ポテンシャルによらずに決まると仮定して、式\ref{4-1}を用いることにしよう。また、全ての親魚は、年をまたぐごとに、1)サブ集団を移動して産卵するか(確率:$m$)、2)同じサブ集団に留まって産卵するか(確率:$s$)、3)死亡するか(確率:$1-m-s$)、のいずれかのイベントを経験するとする。

今回新たに開発した移動に関する関係式について、以下の6種類の親魚に関するその子供(当歳魚)ペアがありえる:
\begin{enumerate}
  \item $N_1(t)\to N_2(t+1)$
  \item $N_1(t)\to N_2(t+2)$
  \item $N_1(t+1)\to N_2(t+2)$
  \item $N_2(t)\to N_1(t+1)$
  \item $N_2(t)\to N_1(t+2)$
  \item $N_2(t+1)\to N_1(t+2)$
\end{enumerate}
右向きの矢印$\to$は、興味ある親魚が年を跨いだ後どのサブ集団に属しているかを表している。1番目を例にとると、$t$年にサブ集団1から、$t+1$年にサブ集団2からそれぞれ当歳魚をサンプルすることを意味する。加えて、同じサブ集団から年を跨いで見つかる半兄弟ペアに基づく関係式についても、以下の6種類が考えられる:
\begin{enumerate}
\setcounter{enumi}{6}
  \item $N_1(t)\to N_1(t+1)$
  \item $N_1(t)\to N_1(t+2)$
  \item $N_1(t+1)\to N_1(t+2)$
  \item $N_2(t)\to N_2(t+1)$
  \item $N_2(t)\to N_2(t+2)$
  \item $N_2(t+1)\to N_2(t+2)$
\end{enumerate}
以上の、合計12種類の式を連立できることを示した。最大14個のパラメータが必要となるのであったが、パラメータ数を応用先に応じて適宜減少させることができる場合、興味あるパラメータを推定することができる。

ちなみに、同一年内に同じサブ集団内で見つかるMHSペアについては、有効集団サイズの情報(正確には有効繁殖サイズ、Effective breeding size)は持っているが、センサスサイズに関するダイレクトな情報は持っていない点に注意。
\end{comment}%%%%%%%%%%%%%%%

\section{Results and Discussion}\label{sec4}

オーダーは$m\approx0.1$程度を想定。これよりもずっと小さければ、遺伝的に異なる系群と考えられFstも大きく出るので、ベイズアスなどから推定できるだろう。一方、その場合はサンプル数が大量になり、今回提案した方法は向かない。

\section{ACKNOWLEDGEMENTS} 

The author thanks MV Bravington for motivating me to pursue this research topic. This work was supported by JSPS KAKENHI Grant Number 19K06862 and 20H03012.


\clearpage

\section*{APPENDIX A}
\setcounter{equation}{0}

\section*{Equation X holds under flexible assumptions for the distribution of surviving offspring number per parent}

\renewcommand{\theequation}{A\arabic{equation}}

これまでは,母親による産卵数の偏りや,父親による放精成功の偏りは考慮していなかった。しかしながら,多回繁殖する魚類の場合,体サイズによって繁殖成功度は大きく変化すること,および体サイズの分布には偏りが存在することから,上記のシステマチックな偏りを考慮する必要があるだろう。例えば,集団に1個体の巨大魚(メス)がいて,その産卵数が他の個体よりも圧倒的に大きい場合,サンプルされる子供の母親はかなりの確率でその巨大魚になるだろう。

サブ集団1(2)にいる親$i(j)$が残す子供の数を$k_{1,i}(k_{2,j})$としよう。$k_{1,i}(k_{2,j})$は期待値$\lambda_{1,i}(\lambda_{2,j})$の分布(例えば負の二項分布など)に従うとしよう。体サイズによる重み付けを考慮したランダム交配を仮定すれば,ペアがMHSである確率は
\begin{align}
\pi | _{\boldsymbol{k_1, k_2}} &= \frac{ 2\sum_{i=1}^M k_{1,i} } { \sum_{i=1}^{N_1} k_{1,i} } \frac{ 2\sum_{j=1}^M k_{2,j} } { \sum_{j=1}^{N_2} k_{2,j} } \frac{1}{M} 
\end{align}
と書くことができる。簡単のため,$\boldsymbol{k_1}=(k_{1,1}, \ldots, k_{1,M}, \ldots, k_{1,N_1})$および$\boldsymbol{k_2}=(k_{1,2}, \ldots, k_{1,M}, \ldots, k_{1,N_2})$と記載し,また親のidが$1$から$M$の個体が移動するとして,idを配置した。$k_{1,i}$は期待値$\lambda_{1,i}$に従う確率変数であることから,このMHSペア確率の平均をとると
\begin{align}
\pi | _{\boldsymbol{\lambda_1, \lambda_2}} &= \mathbb{E}[\pi | _{\boldsymbol{k_1, k_2}}] \nonumber\\
&= \frac{4}{M} \mathbb{E}\left[\frac{ \sum_{i=1}^M k_{1,i} } { \sum_{i=1}^{N_1} k_{1,i} } \frac{ \sum_{j=1}^M k_{2,j} } { \sum_{j=1}^{N_2} k_{2,j} } \right]   \nonumber\\
&\approx \frac{4}{M} \frac{ \mathbb{E}\left[ \sum_{i=1}^M k_{1,i} \sum_{j=1}^M k_{2,j} \right] } {\mathbb{E}\left[ \sum_{i=1}^{N_1} k_{1,i} \sum_{j=1}^{N_2} k_{2,j}\right]} \nonumber\\
&= \frac{4}{M} \frac{ \sum_{i=1}^M \lambda_{1,i} \sum_{j=1}^M \lambda_{2,j} } {\sum_{i=1}^{N_1} \lambda_{1,i} \sum_{j=1}^{N_2} \lambda_{2,j}} 
\end{align}
と近似することができる。留意すべき点として,移動前の産卵数と移動後の産卵数は独立と仮定しているため,式中に$k_i^2$の項が出現せず,結果は期待値$\lambda_i$(産卵ポテンシャルと呼ぶことにしよう)だけに依存する。最初に述べたように,この個体ごとの産卵ポテンシャルはある分布に従うとしよう。例えば,高齢個体ほど体サイズが大きい場合などが想定される。また,この分布はサブ集団1とサブ集団2で異なっていて良い。すなわち,$\lambda_{1,i}$および$\lambda_{2,j}$がそれぞれ独立な確率変数であると考えて,MHSペア確率についてさらに平均をとると
\begin{align}
\pi &= \mathbb{E}[\pi | _{\boldsymbol{\lambda_1, \lambda_2}}] \nonumber\\ 
&\approx \frac{4}{M} \frac{M^2 \mathbb{E}[\lambda_1 \lambda_2]}{N_1 N_2 \mathbb{E}[\lambda_1 \lambda_2]}\nonumber\\
&= \frac{4M}{N_{1}N_{2}}
\label{4-1}
\end{align}
と近似される。結局,繁殖ポテンシャルの不均一性を考慮しない結果(式\ref{3-1})と同様になる。なお,$\boldsymbol{\lambda_1}=(\lambda_{1,1}, \ldots, \lambda_{1,N_1})$および$\boldsymbol{\lambda_2}=(\lambda_{1,2}, \ldots, \lambda_{1,N_2})$。この近似は、独立に決まる確率変数$\lambda_1$と$\lambda_2$について、$\mathbb{E}[\lambda_1/\lambda_2]=\mathbb{E}[\lambda_1] \times \mathbb{E}[1/\lambda_2] \approx 1$となる場合に成立するので、集団中に$\lambda$が極端に低い個体が存在する場合は成立しない(サブ集団1とサブ集団2でセットにして相対的な繁殖ポテンシャルを考えれば良いので、問題ないだろう)。

\subsection{Covariation between migration and reproductive potential}

上のモデルでは,サブ集団1とサブ集団2を構成する親魚の繁殖ポテンシャルの分布をそれぞれ考慮していたが,移動個体の繁殖ポテンシャルの不均一性については考慮していなかった。最後に,移動した親魚の繁殖ポテンシャルについてもある分布に従うと仮定しよう。すなわち,$\lambda_{1,i}, \lambda_{2,j}, \lambda_{M,l}$をそれぞれ独立な確率変数とみなして,MHSペア確率を計算すると,
\begin{align}
\pi | _{\boldsymbol{\lambda_1, \lambda_2, \lambda_M}} &\approx \frac{4}{M} \frac{ \sum_{i=1}^M \lambda_{M,i} \sum_{j=1}^M \lambda_{M,j} } {\left(\sum_{i=1}^{M} \lambda_{M,i} + \sum_{i=M+1}^{N_1} \lambda_{1,i}\right) \left(\sum_{j=1}^{M} \lambda_{M,j} + \sum_{j=M+1}^{N_2} \lambda_{2,j}\right) } 
\end{align}
のようになり,$N_1$, $N_2$が$M$より十分に大きいと仮定すると、この期待値は以下のようになる,
\begin{align}
\pi &\approx \frac{4M}{N_{1}N_{2}} \mathbb{E}\left[ \frac{ \lambda_M^2 }{\lambda_1\lambda_2} \right]
\label{5-1}
\end{align}
すなわち,繁殖ポテンシャルの効果がMHSペア確率に影響するようになる。

Let $k_i$ denote the total number of surviving offspring from a mother  $i$ ($i= 1, 2, \dots, N_{\rm m}$) at time of sampling. Following \cite{Akita_2019} and giving the expected number of the surviving offspring per mother at time of sampling, $\lambda_i$ ($>0$), $k_i$ is assumed to follow a negative binomial distribution through a conventional parametrization:
\begin{equation}
{\rm Pr}[ k_i | \lambda_i ] = \frac{\Gamma[k_i+\phi]}{k_i! \Gamma[\phi]} \left( \frac{\lambda_i}{\phi+\lambda_i} \right)^{k_i} \left( \frac{\phi}{\phi+\lambda_i} \right)^{\phi}, 
\label{NB}
\end{equation}
where $\phi$ ($>0$) denotes the overdispersion parameter describing the degree of nonparental variation. In the current work, $\phi$ is assumed to be constant across mothers, whereas the expected number of surviving offspring ($\lambda_i$) varies across mothers. The mean and variance of this distribution are denoted by $\lambda_i$ and $\lambda_i + \lambda_i^2/\phi$, respectively. In the limit of infinite $\phi$, this distribution becomes a Poisson distribution, which is given by ${\rm Pr}[ k_i | \lambda_i ]=\lambda_i^{k_i} \mathrm{e}^{-\lambda_i}/(k_i!)$. We assume $\lambda_i$ to be independent and identically distributed with the density function $f(\lambda)$, producing a parental variation. The shape of the density function is often complex, but may be described by information from the population, for example, the mother's weight composition in the population. The specific form of $f(\lambda)$ is given in {\bf Appendix B} and is used for running an individual-based model.  

The variance of the offspring number can be given by 
\begin{eqnarray}
\mathbb{V}[k] &=& \mathbb{E}[ \mathbb{V}[k|\lambda] ] + \mathbb{V}[ \mathbb{E}[k|\lambda] ] \nonumber\\
&=& \mathbb{E}[ \lambda + \lambda^2 / \phi ] + \mathbb{V}[ \lambda ].
\label{V_k}
\end{eqnarray}



\begin{comment}
Many authors submitting \( \sin \cos \tan \inf_{x} \) to NJD journals use \LaTeXe\ to
prepare their papers. This paper describes the
\textsf{WileyNJD-v2.cls} class file which can be used to convert
articles produced with other \LaTeXe\ class files into the correct
form for publication in \emph{Wiley NJD Journals}.

The \textsf{WileyNJD-v2.cls} class file preserves much of the standard
\LaTeXe\ interface so that any document which was produced using
the standard \LaTeXe\ \textsf{article} style can easily be
converted to work with the \textsf{WileyNJD-v2} style. However, the
width of text and typesize will vary from that of
\textsf{article.cls}; therefore, \emph{line breaks will change}
and it is likely that displayed mathematics and tabular material
will need re-setting.

In the following sections we describe how to lay out your code to
use \textsf{WileyNJD-v2.cls} to reproduce the typographical look of
\emph{Wiley NJD Journals}.

\subsection{Procedure to install fonts}
 
\begin{enumerate}
\item All font files are available under the Stix-fonts folder 
\item Font installer is available under the same folder Windows-Stix-fontinstaller.exe
\item Execute (double click the EXE file) the EXE file that will install all fonts/map files to your local drive.
\end{enumerate}

\subsection{The Three Golden Rules} 

Before we proceed, we would like to
stress \emph{three golden rules} that need to be followed to
enable the most efficient use of your code at the typesetting
stage:
\begin{enumerate}
\item[(i)] keep your own macros to an absolute minimum;

\item[(ii)] as \TeX\ is designed to make sensible spacing
decisions by itself, do \emph{not} use explicit horizontal or
vertical spacing commands, except in a few accepted (mostly
mathematical) situations, such as \verb"\," before a
differential~d, or \verb"\quad" to separate an equation from its
qualifier;

\item[(iii)] follow the \emph{NJD} reference style.
\begin{enumerate}[a.]
\item Chemistry --- Use the ``AMA'' option as \verb"\documentclass[AMA]{WileyNJD-v2.cls}"

\end{enumerate}
\end{enumerate}


\section{Getting Started} The \textsf{WileyNJD-v2.cls} class file should run
on any standard \LaTeXe\ installation. If any of the fonts, class
files or packages it requires are missing from your installation,
they can be found on the \emph{\TeX\ Live} CD-ROMs or from CTAN.

LaTeX document class options

\begin{flushleft}
\begin{enumerate}[a.]
\item STIX1COL--- For STIX font large one column layout use the ``STIX1COL'' option as \verb"\documentclass[AMA,STIX1COL]{WileyNJD-v2}"
\item STIX2COL--- For STIX font large two column layout use the ``STIX2COL'' option as \verb"\documentclass[AMA,STIX2COL]{WileyNJD-v2}"
\item STIXSMALL--- For STIX font small layout use the ``STIXSMALL'' option as \verb"\documentclass[AMA,STIXSMALL]{WileyNJD-v2}"
\end{enumerate}
\end{flushleft}



\section{The Article Header Information}
The heading for any file using \textsf{WileyNJD-v2.cls} is shown in
Figure~\ref{F1}.

\subsection{Remarks}

\begin{enumerate}[(I).]

\item Use \verb"\title{<title> \protect\thanks{<title footnotes}}" for article title and title footnote.
\item Use \verb"\authormark{}" for running heads.

\item Note the use of \verb"\author[<link>]{<name>}" and \verb"\address[<link>]{<name>}" to
link names and addresses. The author for correspondence is marked
by ``*'' and \verb"\corres{}" is used to give that
author's address, which will be printed besides abstract, prefaced by
`Correspondence to:'.

\item For submitting a double-spaced manuscript, add
\verb"doublespace" as an option to the documentclass line. \verb"\documentclass[doublespace]{WileyNJD-v2}"

\item Use \verb"\presentaddress{}" for present address.

\item In abstract \verb"\abstract[<title>]{abstract paragraph}" use optional parameter for title followed by abstract paragraph.

\item For Key words use \verb"\keywords{}".

\item For how to site use \verb"\jnlcitation{\cname{\author{<author name>},"

\verb"\ctitle{<title>}, \cjournal{<Journal name>}, \cvol{<vol>}.}}".

\item For title page abbreviations use \verb"\footnotetext{<\textbf{Abbreviation title:} Abreviations>}"

\item Use \verb"\articletype{<article category}" for article header information

\item Use \verb"\received{<received date>} \revised{<revised date>} \accepted{<accepted date>}" for history dates.

\end{enumerate}



\begin{figure}[p]
\setlength{\fboxsep}{0pt}%
\setlength{\fboxrule}{0pt}%
\begin{center}
\begin{boxedverbatim}

\documentclass[AMA,STIX1COL]{WileyNJD-v2}

\articletype{Article Type}%

\received{26 April 2016}
\revised{6 June 2016}
\accepted{6 June 2016}

\begin{document}

\title{<Initial cap, lower case>\protect\thanks{<title footnote.>}}

\author[<address link>]{<Author name><corresponding author*>}

\author[<address link>,<address link>]{Author Name}

\authormark{AUTHOR ONE \textsc{et al}}

\address[<address link>]{\orgdiv{<Org Division>}, \orgname{<Org name>}, 
\orgaddress{\state{<State name>}, \country{<Country name>}}}
\address[<address link>]{\orgdiv{<Org Division>}, \orgname{<Org name>},
 \orgaddress{\state{<State name>}, \country{<Country name>}}}

\corres{<corresponding author link*> <author name, address.
 \email{<authorone@email.com>}}

\presentaddress{<Present address>}

\abstract[<Abstract heading>]{<Abstract paragraph>}

\keywords{<keyword1>, <keyword2>,...}

\jnlcitation{\cname{%
\author{<aurhor name>}, 
\author{<aurhor name>}, 
\author{<aurhor name>}, 
\author{<aurhor name>},  and 
\author{<aurhor name>}} (\cyear{<year>}), 
\ctitle{<journal title>}, \cjournal{<journal name>} <year> <vol> Page <xxx>-<xxx>}

\footnotetext{\textbf{<abbreviation head:>} <abbreviations> ..}

\maketitle

\section{Introduction}
.
.
.
\end{boxedverbatim}
\end{center}
\caption{Example for title page.\label{F1}}
\end{figure}

\section{The Body of the Article}

\subsection{Section headings}

\begin{enumerate}[(H1)]
\item Section --- use \verb"\section{}"
\item SubSection--- use \verb"\subsection{}"
\item SubSubSectioin--- use \verb"\subsubsection{}"
\item Paragraph--- use \verb"\paragraph{}"
\item Subparagraph--- use \verb"\subparagraph{}"
\end{enumerate}

\subsection{Mathematics} \textsf{WileyNJD-v2.cls} makes the full
functionality of \AmS\/\TeX\ available. We encourage the use of
the \verb"align", \verb"gather" and \verb"multline" environments
for displayed mathematics.

\subsection{Figures and Tables}

\textsf{WileyNJD-v2.cls} uses the
\textsf{graphicx} package for handling figures.

Figures are called in as follows:
\begin{verbatim}
\begin{figure}
\centering
\includegraphics{<figure name>}
\caption{<Figure caption>}
\end{figure}
\end{verbatim}

The standard coding for a table is shown in Figure~\ref{F2}.

\begin{figure}[h]
\setlength{\fboxsep}{0pt}%
\setlength{\fboxrule}{0pt}%
\begin{center}
\begin{boxedverbatim}
\begin{table}
\caption{<Table caption>}
\centering
\begin{tabular}{<table alignment>}
\toprule
<column headings>\\
\midrule
<table entries
(separated by & as usual)>\\
<table entries>\\
.
.
.\\
\bottomrule
\end{tabular}
\begin{tablenotes}
\item Source: xxx.
\item[1] xxx.
\item[2] xxx.
\end{tablenotes}
\end{table}
\end{boxedverbatim}
\end{center}
\caption{Example for table layout.\label{F2}}
\end{figure}

\subsection{Cross-referencing}
The use of the \LaTeX\ cross-reference system
for figures, tables, equations, etc., is encouraged
(using \verb"\ref{<name>}" and \verb"\label{<name>}").

\subsection{Box text}

\begin{verbatim}
\begin{boxtext}
\section*{<title>}%
Paragraph
\end{boxtext}
\end{verbatim}

\subsection{List items}

\subsubsection{Enumerate list styles}
\begin{verbatim}
\begin{enumerate}[1]
\item 
\end{enumerate}

\begin{enumerate}[1.]
\item 
\end{enumerate}

\begin{enumerate}[(1)]
\item 
\end{enumerate}

\begin{enumerate}[I]
\item 
\end{enumerate}

\begin{enumerate}[i]
\item 
\end{enumerate}

\begin{enumerate}[a]
\item 
\end{enumerate}

\end{verbatim}

\subsubsection{Bullet list styles}

\begin{verbatim}
\begin{itemize}
\item 
\end{itemize}
\end{verbatim}

\subsubsection{Description list}

\begin{verbatim}
\begin{description}
\item[<entry>] description text.
\end{description}
\end{verbatim}

\subsection{Enunciations}

\begin{verbatim}
\begin{theorem}[<Theorem subhead>]\label{thm1}
<theorem text>. 
\end{theorem}

\begin{proposition}[<proposition subhead>]\label{pro1}
<proposition text>. 
\end{proposition}

\begin{definition}[<definition subhead>]\label{dfn1}
<definition text>. 
\end{definition}

\begin{proof}
<proof text>. 
\end{proof}

\begin{proof}[Proof of Theorem~\ref{thm1}]
<proof text>.
\end{proof}

\end{verbatim}

\subsection{Program codes}

Use \verb+\begin{verbatim}...\end{verbatim}+ for program codes without math. Use \verb+\begin{alltt}...\end{alltt}+ for program codes with math. Based on the text provided inside the optional argument of \verb+\begin{code}[Psecode|Listing|Box|Code|+\hfill\break \verb+Specification|Procedure|Sourcecode|Program]...+ \verb+\end{code}+ tag corresponding boxed like floats are generated. Also note that \verb+\begin{code}[Code|Listing]...+ \verb+\end{code}+ tag with either Code or Listing text as optional argument text are set with computer modern typewriter font.  All other code environments are set with normal text font. Refer below example:

\begin{verbatim}
\begin{lstlisting}[caption={Descriptive Caption Text},label=DescriptiveLabel]
for i:=maxint to 0 do
begin
{ do nothing }
end;
Write('Case insensitive ');
WritE('Pascal keywords.');
\end{lstlisting}
\end{verbatim}


\subsection{Acknowledgements} 

This work was supported by JSPS KAKENHI Grant Number 19K06862 and 20H03012.
An Acknowledgements section is started with \verb"\ack" or
\verb"\acks" for \textit{Acknowledgement} or
\textit{Acknowledgements}, respectively. It must be placed just
before the References.

\subsection{Bibliography}

\begin{enumerate}[1]
\item Use \verb"\bibliography{wileyNJD-AMA}" BST file for AMA reference style
\item Use \verb"\bibliography{wileyNJD-APA}" BST file for APA reference style
\item Use \verb"\bibliography{wileyNJD-AMS}" BST file for AMS reference style
\item Use \verb"\bibliography{wileyNJD-VANCOUVER}" BST file for Vancouver reference style
\item Use \verb"\bibliography{wileyNJD-ACS}" BST file for Chemistry reference style
\end{enumerate}

The normal commands for producing the reference list are:

\begin{verbatim}
\begin{thebibliography}{99}
\bibitem{<x-ref label>}
         <Reference details>
.
.
.
\end{thebibliography}
\end{verbatim}

\subsection{Appendix Section}

\begin{verbatim}
\appendix

\section{Section title of first appendix\label{app1}}
.
.
.

\end{verbatim}
\end{comment}

\end{document}
