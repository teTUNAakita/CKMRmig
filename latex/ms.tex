\documentclass[12pt, English]{article}
\usepackage{geometry}
\geometry{verbose,letterpaper,tmargin=2.54cm,bmargin=2.54cm,lmargin=2.54cm,rmargin=2.54cm}
%\usepackage{apacite}
\usepackage[natbibapa]{apacite}
%\renewcommand{\APACrefYearMonthDay}[3]{\APACrefYear{#1} 
%  \cite<pre>[post]{key}
\usepackage{amsfonts}
\usepackage{amssymb}
\usepackage{amstext}
\usepackage{amsmath}
\usepackage{comment}
\usepackage{enumerate}
\usepackage{graphicx}
\usepackage{boxedminipage}
\usepackage{oldgerm}
\usepackage{setspace}
\usepackage{booktabs}
\usepackage{wrapfig}
\usepackage{indentfirst}
\usepackage{mathptmx}
\usepackage{setspace}
\usepackage{lscape} 
\usepackage[mathlines]{lineno}
\renewcommand{\ttdefault}{mathptmx}
\usepackage[latin9]{inputenc}
\usepackage{bm}
\usepackage{relsize}
\usepackage[normalem]{ulem} %%
%\linenumbers
\usepackage{bm}
\setlength{\parindent}{20pt}
\usepackage[labelformat=empty,labelsep=none]{caption}
\usepackage[singlelinecheck=false]{caption}
\usepackage{color}
\def\RED#1{\textcolor{red}{#1}} %% 
\def\BLUE#1{\textcolor{blue}{#1}} %% 
\begin{document}

\title{Estimating contemporary migration numbers of adults for iteroparous species using kinship pairs found between populations}
\author{
Tetsuya Akita$^{a}$\thanks{
\emph{Corresponding author}:
    Fisheries Resources Institute, Fisheries Research and Education Agency, Kanagawa 236-8648, Japan.
    E-mail: akitatetsuya1981@affrc.go.jp} \\ 
   \small $^{a}$Fisheries Resources Institute, Fisheries Research and Education Agency, \\
   \small Kanagawa, 236-8648, Japan.
   }
\date{Manuscript Intended for \emph{Methods in Ecology and Evolution}, \today}
\maketitle

\begin{abstract}
本ノートでは,多回繁殖(iteroparity)する種を対象として,半兄弟ペア数から親魚による2つのサブ集団間の移動率を推定する理論($\hat{M}=N_1N_2H_{\rm obs}/(4n_1n_2)$)を紹介する。この推定式は、親をサンプルする通常の標識放流と同じである。もし、繁殖ポテンシャルの分布が集団間で異なっていても上記の式で問題ないことを示した。一方,サンプル数としては$\sqrt {N}$以上のオーダーが求められる。
\\
\end{abstract}
\textbf{Key words:} effective population size; sibship assignment; overdispersed reproduction; close-kin mark-recapture\\
\textbf{Running head:} Estimating migration rates using kinship pairs\\

\pagebreak
\begin{spacing}{1.9}
%%%%%%%%%%%%%%%%%%%%%%%%%%%%%%%%%%%%%%%%%%%%%%%%%%
%%%FIGURE S1%%%
\newcommand{\NeNbias}{S1}
%%%%%%%%%%%

%%%FIGURE S2%%%
\newcommand{\Nebias}{S2}
%%%%%%%%%%%

%%%FIGURE S3%%%
\newcommand{\Ninvbias}{S3}
%%%%%%%%%%%

%%%FIGURE S4%%%
\newcommand{\NeNcv}{S4}
%%%%%%%%%%%

%%%FIGURE S5%%%
\newcommand{\Necv}{S5}
%%%%%%%%%%%

%%%FIGURE S6%%%
\newcommand{\Ninvcv}{S6}
%%%%%%%%%%%

\section{INTRODUCTION}

% 1: Estimating contemporary migration is important for both ecologically and evolutionally reasons

In conservation and wildlife management, estimating the current ecological information is essential for monitoring population levels and proposing management strategies. Contemporary migration rate, the recent rate of movement of individuals or gametes between pre-defined populations, is a critical component to determine the current degree of gene flow and synchrony among the population \cite[]{Waples_2006_MolEcol, https://doi.org/10.1111/j.1365-294X.2010.04688.x}. Therefore, estimation of contemporary migration rate provides information about the degree of genetic differentiation and demographic dependency, which can delineate evolutionarily significant units and identify appropriate management units \cite[]{moritz1994defining, PALSBOLL200711}. In addition, the availability of migration rate can link to underlying population dynamics, allowing for a more precise and flexible evaluation of a management for many relevant topics including stock assessment in fisheries \cite[]{methot2013stock} and the invasive species control \cite[]{doi:10.1146/annurev.ecolsys.32.081501.114037}. 

% 2: Mark recapture method can handle ecological scale but may miss the small migration rate

There are essentially two approaches for estimating the contemporary migration rate. The first is to use the mark-recapture (MR) method that estimates migration rate along with other population parameters \cite[]{kery2011bayesian, Thorson_2021}. The rational is that, for example, conventional tags provide release and recovery location at known release and recovery dates, generating a movement fraction matrix among strata per given time interval. While this direct method is relatively easy to interpret, it is generally hampered by several sources of uncertainties, e.g., tag loss, tagging-related mortality, and reporting rate \cite[]{hilborn2013quantitative}. In addition, the MR method for estimating migration rate is generally limited in practice to situations where collection of a sufficient number of adult sample at multiple sites in a short period is available. 

% 3: Population genetics method can handle evolutionally scale but may miss the large migration rate

The second is to use numerous genetic markers to assign individuals to source populations, allowing the inference of recent migration \cite[since][]{https://doi.org/10.1111/j.1365-294X.1995.tb00227.x}. The model of \cite{Wilson_2003}, which is one of such population assignment methods and implemented in a software `BayesAss', can output the point estimate as an element of a current migration matrix. However, as evaluated by several authors \cite[]{https://doi.org/10.1111/j.1365-294X.2007.03218.x, https://doi.org/10.1111/mec.12806}, the accuracy of estimated migration rates by the model is valid only when populations are highly differentiated ($F_{\rm ST}\ge0.05$), suggesting that it may be difficult to infer the degree of demographic dependency among populations since demographic independency is realized even if there are many migrants. 

% 4: Wang 2014 can handle ecological scale with realistic situations but limited to offspring movement

Parental assignments, which is based on genetic markers and often used as a complement to the population assignment method in ecological studies, also provide information about the current level of migration. In contrast to the population assignment method, the parental assignment method does not require the population differentiation \cite[]{https://doi.org/10.1111/mec.12806}. However, the current methodology focuses on estimating dispersal kernels and strongly depends on the assumption adults are not migrant, e.g., pollen/seeds dispersal in plants \cite[]{doi:10.1080/07352689.2010.481167} or larval drift in marine animals \cite[]{https://doi.org/10.1111/eva.12288}; thus, it does not apply the movement of iteroparous species (i.e., multiple reproductive cycles during the lifetime) that may change the spawning ground for each breeding season. 

% 5: CKMR can cover the range that Mark recapture can handle with more realistic situations

Close-kin mark-recapture (CKMR) is a recently developed method for estimating adult population size along with life-history parameters that utilizes the known kinship information in a sample \cite[reviewed in][]{https://doi.org/10.1111/faf.12615}. The rationale is that the presence of a kinship pair in the sample is analogous to the recapture of a marked individual in MR method. Kinship pairs in the sample are less likely to be observed in larger populations; thus, the number of kinship pairs may reflect an adult number in the population \cite[]{bravington2016close}. While recent studies of CKMR argued that availability of kinship information in estimating the migration rate \cite[]{https://doi.org/10.1002/ece3.6296, 10.1093/icesjms/fsac002}, the CKMR-based migration estimation has not been well developed and rigorously examined in a simulation study. 

% 6: In this paper, ~

In this study, we propose a new method for estimating the contemporary migration number or rate of adults between the two pre-defined populations in iteroparous species, where the migration direction is specified. Assuming that kinships are genetically detected without any error, this method is based on the numbers of half-sibling (HS) and parent-offspring (PO) pairs in a sample. Sampling can be invasive or non-invasive and is completed at two breeding seasons; sampling offspring (young-of-year individuals) and parents probably shares PO relationship with sampled offspring in the one population at the first breeding season and in the another population at the second breeding season. The rationale is that the number of PO and HS pairs found between populations have information as to how often parental movements are realized. Our model explicitly incorporates reproductive variation within and between populations, making it possible to target a species whose fertility is affected by environmental differences between populations. 

First, we explain the modeling assumption and sampling scheme. Then, we analytically determine the estimators of the contemporary migration number or rate of adults, which are based on the numbers of HS and/or PO pairs. Finally, by running an individual-based model, we investigate the performance of the estimator and provide a guide for a sample size. It is noteworthy that our modeling framework can be applied to diverse animal species. However, the description of the model focuses on fish species, which are presently the best candidate target of our proposed method.

\section{Theory}\label{sec2}
In this section, we present the theoretical foundation for estimating a migrant number and a migration rate of iteroparous species using HS or PO pairs, which are found between two populations with a different sample timing. While the estimators can hold under flexible assumptions for reproduction (see {\bf Appendix 2}), here we assume a relatively simple situation for explanation purposes. The main symbols used in the current paper are summarized in {\bf Table \ref{symbols}}. 

%%%%%%%%%%%%%%
\begin{center}
\fbox{Table \ref{symbols} is here. }
\end{center}
%%%%%%%%%%%%%%

\subsection{Hypothetical population}

Suppose that we have a set of two populations labelled 1 and 2 where random mating occurs within each population and that parents can move to the other population after the end of the reproductive season. Hereafter, without loss of generality, we consider the movement of parents in the direction from population 1 to population 2 and focus on the migration number or rate, and that is what we attempt to estimate. In this framework, we show that considering two reproductive seasons (hereafter, we call them "the first year" and "the second year") is enough to estimate it. {\bf Figure \ref{cartoon}a} presents a schematic representation of the kinship relationships and movements of parents. 

Suppose there are $N_1$ parents in the population 1 at the beginning of the first year. Each parent produces offspring in which the number is governed by the parent's reproductive potential, denoted by $\lambda_{i,1}$ ($i=1,2,\ldots,N_1$). After the reproductive season, part of the parents would start to moving towards the population 2; $M$ survived migrants arrive in the population 2 so that there are $N_2$ parents in the population 2 at the beginning of the second year ($M \le N_1$ and $M \le N_2$). Similar to the population 1, $N_2$ parents produce offspring in which the number is governed by $\lambda_{j,2}$ ($j=1,2,\ldots,N_2$). The reproductive potential is determined by several factors and its details are summarized in {\bf Appendix 1}; theoretically speaking, the number of offspring for each parent is set as a random variable with mean $\lambda_{i,1}$ or $\lambda_{j,2}$. 

%%%%%%%%%%%%%%
\begin{center}
\fbox{Figure \ref{cartoon} is here. }
\end{center}
%%%%%%%%%%%%%%

\subsection{Sampling}

To estimate the migration number or rate, we utilize the number of HS and/or PO pairs observed in a sample. In the both populations, $n_{\rm O, 1}$ and $n_{\rm O, 2}$ offspring are randomly sampled in the first and second year, respectively, which are identified as young-of-year individuals without errors. Additionally, $n_{\rm P, 1}$ and $n_{\rm P, 2}$ parents are randomly sampled immediately after the end of the reproductive season in the first and second year, respectively. For mathematical tractability, parents have to survive during the reproductive season; therefore, both mother and father of a certain offspring have a potential to be sampled even in the same year. It should be noted that all the four kinds of a sample (i.e., $n_{\rm P, 1}$, $n_{\rm P, 2}$, $n_{\rm O, 1}$, and $n_{\rm O, 2}$) are not always required for estimating migration number or rate; what kinds of a sample are needed depends on the situation, as explained later. Up to the subsection 2.6, we only focus HS pairs and PO pairs found between the two populations. For the applicability of fishery management, we assume invasive sampling procedure, preventing the finding of a parent-offspring pair such that a parent and offspring is sampled in the population 1 and 2, respectively. {\bf Figure~\ref{cartoon}b} illustrates a timeline of sampling scheme for hypothetical populations. 

In the example depicted in {\bf Fig~\ref{cartoon}a}, five offspring and four parents are sampled in the population 1 and four offspring and six parents are sampled in the population 2, and two HS and one PO pairs found between the two populations are observed. In addition, there are several PO pairs observed within the same population, which are available to estimate parent numbers, as explained in the following section. In our modeling framework, if full-sibling (FS) relationship is found, we count it as two HS pairs. The numbers of HS and PO pairs found between the two populations are determined by pairwise comparison of all the sample individuals ($n_{\rm O,1}\times n_{\rm O,2}$ and $n_{\rm O,1}\times n_{\rm P,2}$ comparisons, respectively). 

\subsection{HS pair-based model}

Here, we consider the probability that two offspring sampled in the population 1 and 2 share an HS relationship with an arbitrary mother or father, denoted by $\pi_{\rm HS, bet}$. For the explanation purpose, we assume equal reproductive potential among parents up to the subsection 2.7, although this assumption can be relaxed for most of the case (see details in {\bf Appendix 2}). $\pi_{\rm HS, bet}$ can be partitioned into the three probabilities: (i) the probability that sampled offspring in the population 1 born to a parent that safely arrives in the population 2 (hereafter, called "migrant"); (ii) the probability that sampled offspring in the population 2 born to a migrant; and (iii) the probability that the migrant of a sampled offspring in the population 1 and the migrant of a sampled offspring in the population 2 are identical. 

To assess these probabilities, recall there must be two parents of arbitral offspring; the first probability is the sum of father-offspring and mother-offspring relationship, which can be written by $r_{\rm M}M/(r_{\rm S,1}N_{1}) + (1 - r_{\rm M})M/((1-r_{\rm S,1})N_{1})$, where $r_{\rm M}$ and $r_{\rm S,1}$ indicates sex ratio of parents in the migrants and in the whole of the population 1, respectively. Assuming $r_{\rm M} = r_{\rm S,1}$, the probability is simplified as $2M/N_{1}$. Similarly, the second probability is simplified as $2M/N_{2}$, implying the assumption of equal sex ratio in the migrants and the others. The third probability corresponds to the probability that randomly selected two migrants are identical, that is $1/M$. Taken together, we obtain
\begin{align}
\pi_{\rm HS,bet} &= \frac{2M}{N_{1}} \frac{2M}{N_{2}} \frac{1}{M} \nonumber\\
&= \frac{4M}{N_{1}N_{2}}.
\label{HS-1}
\end{align}
This form holds under a flexible setting for the reproductive potential, such as a situation with random variable of $\lambda_{i,1}$ and $\lambda_{j,2}$, as noted in {\bf Appendix 2}. If all $N_1$ parents safely move to the population 2 (i.e., $M = N_1$), $\pi_{\rm HS,bet}$ equals $4/N_2$; meanwhile, if the population 2 consists only of migrants (i.e., $M = N_2$), $\pi_{\rm HS,bet}$ equals $4/N_1$. These probabilities in the extreme cases are similar forms to HS probability that is randomly sampled from two different cohorts within a population \cite[]{bravington2016close}. 

Let $H_{\rm HS,bet}$ be the number of HS pairs found in offspring samples of size $n_{\rm O, 1}$ and $n_{\rm O, 2}$. Assuming that the total number of HS pairs in the population is much higher than $H_{\rm HS,bet}$, the distribution is approximated by a binomial form (i.e., $H_{\rm HS,bet} \sim {\rm Binom}[\pi_{\rm HS,bet}, n_{\rm O, 1} n_{\rm O, 2}]$); thus, the theoretical expectation of $H_{\rm HS,bet}$ is

\begin{align}
\mathbb{E}[H_{\rm HS, bet}] &= \pi_{\rm HS,bet} n_{\rm O, 1} n_{\rm O, 2} \nonumber\\
&= \frac{ 4 n_{\rm O, 1} n_{\rm O, 2} M } { N_{1} N_{2} } \label{HS-2-1}\\
&= \frac{ 4 n_{\rm O, 1} n_{\rm O, 2} m } { N_{2} },
\label{HS-2-2}
\end{align}
where $m$ indicates the migration rate satisfying $M = mN_{1}$. The observed number of HS pairs in a sample found between the two populations is defined by $\widetilde{H}_{\rm HS, bet}$, and $\mathbb{E}[H_{\rm HS, bet}]$ in Equation \ref{HS-2-1} is replaced by $\widetilde{H}_{\rm HS, bet}$, generating the linear estimator of $M$:

\begin{align}
\widehat{M_1} &= \frac{N_{1} N_{2} \widetilde{H}_{\rm HS, bet} } {4 n_{\rm O, 1} n_{\rm O, 2} }.
\label{HS-3}
\end{align}
In this paper, a ``tilde'' and ``hat'' indicates the observation and estimator of a variable, respectively. Similarly, $\mathbb{E}[H_{\rm HS, bet}]$ in Equation \ref{HS-2-2} is replaced by $\widetilde{H}_{\rm HS, bet}$, generating the linear estimator of $m$:

\begin{align}
\widehat{m_1} &= \frac{N_{2} \widetilde{H}_{\rm HS, bet} } {4 n_{\rm O, 1} n_{\rm O, 2} }
\label{HS-4}
\end{align}
Subscription of these estimators indicates the numbering of proposed estimators appeared in this paper, which is summarized in Table \ref{parameter}. 

%%%%%%%%%%%%%%
\begin{center}
\fbox{Table \ref{parameter} is here. }
\end{center}
%%%%%%%%%%%%%%

\subsection{PO pair-based model}

Next, we consider the probability that offspring sampled in the population 1 and a parent sampled in the population 2 share an PO relationship, denoted by $\pi_{\rm PO, bet}$. $\pi_{\rm PO, bet}$ can be partitioned into the three probabilities: (i) the probability that sampled offspring in the population 1 born to a migrant; (ii) the probability that a migrant is sampled; and (iii) the probability that the migrant of a sampled offspring in the population 1 and the migrant sampled in the population 2 are identical. The first and second probabilities are the same as the one introduced in the previous subsection, that is, $2M/N_{1}$ and $1/M$. The second probability is $M/N_{2}$ by definition. Taken together, we obtain
\begin{align}
\pi_{\rm PO,bet} &= \frac{2M}{N_{1}} \frac{M}{N_{2}} \frac{1}{M} \nonumber\\
&= \frac{2M}{N_{1}N_{2}}.
\label{PO-1}
\end{align}
This form also holds under a flexible setting for the reproductive potential, as noted in {\bf Appendix}. If all $N_1$ parents safely move to the population 2 (i.e., $M = N_1$), $\pi_{\rm PO,bet}$ equals $2/N_2$; meanwhile, if the population 2 consists only of migrants (i.e., $M = N_2$), $\pi_{\rm PO,bet}$ equals $2/N_1$. These probabilities in the extreme cases are similar forms to PO probability that is randomly sampled within a population \cite[]{bravington2016close}. 

Let $H_{\rm PO,bet}$ be the number of PO pairs found in offspring samples of size $n_{\rm O, 1}$ and parent samples of size $n_{\rm P, 2}$. Assuming that the total number of PO pairs in the two populations is much higher than $H_{\rm PO,bet}$, the distribution is approximated by a binomial form (i.e., $H_{\rm PO,bet} \sim {\rm Binom}[\pi_{\rm PO,bet}, n_{\rm O, 1} n_{\rm P, 2}]$); thus, the theoretical expectation of $H_{\rm PO,bet}$ is
\begin{align}
\mathbb{E}[H_{\rm PO, bet}] &= \pi_{\rm PO,bet} n_{\rm O, 1} n_{\rm P, 2} \nonumber\\
&= \frac{ 2 n_{\rm O, 1} n_{\rm P, 2} M } { N_{1} N_{2} } \label{PO-2-1} \\
&= \frac{ 2 n_{\rm O, 1} n_{\rm P, 2} m } { N_{2} }.
\label{PO-2-2}
\end{align}
The observed number of PO pairs in a sample is defined by $\widetilde{H}_{\rm PO, bet}$, and $\mathbb{E}[H_{\rm PO, bet}]$ in Equations \ref{PO-2-1} and \ref{PO-2-2} are replaced by $\widetilde{H}_{\rm PO, bet}$, generating the linear estimator of $M$ and $m$:
\begin{align}
\widehat{M_2} &= \frac{N_{1} N_{2} \widetilde{H}_{\rm PO,bet} } {2 n_{\rm O, 1} n_{\rm P, 2} },
\label{PO-3}
\end{align}
and
\begin{align}
\widehat{m_2} &= \frac{N_{2} \widetilde{H}_{\rm PO,bet} } {2 n_{\rm O, 1} n_{\rm P, 2} }.
\label{PO-4}
\end{align}

\subsection{Required sample size}

Proposed estimators are based on the observed number of kinship pairs. Their expected number is linearly determined by the number of pairwise comparison (Eqs.~\ref{HS-2-1}-\ref{HS-2-2} and \ref{PO-2-1}-\ref{PO-2-2}), providing guidance for a sample size to ensure the condition that at least one or more kinship pairs can be found. The conditions are given by
\begin{align}
n_{\rm O, 1} n_{\rm O, 2} &> \frac{ N_{2} } { 4m },
\label{RS-1}
\end{align}
and
\begin{align}
n_{\rm O, 1} n_{\rm P, 2} &> \frac{ N_{2} } { 2m }.
\label{RS-2}
\end{align}
Roughly speaking, required sample size to ensure above conditions is that $n>\sqrt{N_2}$, where $n = n_{\rm O, 1}=n_{\rm O, 2}=n_{\rm P, 2}$. It should be noted that very tiny $m$, which dramatically increases the required sample size, produces significant genetic differences between two populations. In such a case, there are several methods to estimate migration rate by population genetics technique; alternatively, we focus the situation of moderate/large migration rate, which leads little genetic differences between two populations (see details in {\bf Discussion} section). 

Other guidance for the required sample size is obtained by an approximate lower bound on the CV (coefficient of variation) of $M_1$ or $m_1$, defined by $1/\sqrt{H_{\rm HS, bet}}$, which is applied in the context of a classic mark-recapture \cite[]{seber1982estimation} or close-kin mark-recapture \cite[]{bravington2016close}. For example, to achieve a 30\% CV, the target of $H_{\rm HS, bet}$ is more than 10, providing the required sample size when population parameters are given (e.g., $N_1$, $N_2$ and $M$). Similarly, the CV of $M_2$ or $m_2$, defined by $1/\sqrt{H_{\rm PO, bet}}$, provides the required sample size for estimating $M_2$ or $m_2$. 

\subsection{Efficient use of kinship pairs found between populations}

When the both HS pairs and PO pairs, which are found between populations, are simultaneously available, we can obtain efficient estimators by combined $\widehat{M_1}$ and $\widehat{M_2}$ for migration number: 
\begin{align}
\widehat{M_3} &= \frac{ N_{1} N_{2} \left(\widetilde{H}_{\rm HS,bet} + \widetilde{H}_{\rm PO,bet}\right) } {2n_{\rm O, 1} \left(2 n_{\rm O, 2} + n_{\rm P, 2}\right) },
\label{AP-1}
\end{align}
and by combining $\widehat{m_1}$ and $\widehat{m_2}$ for migration rate: 
\begin{align}
\widehat{m_3} &= \frac{ N_{2} \left(\widetilde{H}_{\rm HS,bet} + \widetilde{H}_{\rm PO,bet}\right) } {2n_{\rm O, 1} \left(2 n_{\rm O, 2} + n_{\rm P, 2}\right)}.
\label{AP-2}
\end{align}
For those estimators, $\widetilde{H}_{\rm HS,bet}$ and $\widetilde{H}_{\rm PO,bet}$ is weighted by a sample size $n_{\rm O, 2}$ and $n_{\rm P, 2}$, respectively. 

\subsection{Estimation of parent number by PO pairs found within a population}

So far, we provide the formulation of the estimators, presented in Equations \ref{HS-3}-\ref{HS-4} and \ref{PO-3}-\ref{PO-4}, is a function with parent numbers for each population ($N_1$ and/or $N_2$); in other words, such estimators are available only when those parent number(s) are known. In this subsection, we provide the treatment for estimating the unknown parent numbers by additional usage of PO pairs, producing estimators for migrant number and migration rate that can be obtained only from the genetic data. 

When PO pairs found in offspring samples and parent samples both from the population 2 in the second year are available, standard estimator of parental number can be obtained \cite[]{bravington2016close}, given by

\begin{align}
\widehat{N_2} &= \frac{ 2 n_{\rm O, 2} n_{\rm P, 2} + 1 }{\widetilde{H}_{\rm PO, 2} + 1} ,
\label{AP-3}
\end{align}
where $\widetilde{H}_{\rm PO, 2}$ is the observed number of PO pairs found in offspring samples of size $n_{\rm O, 2}$ and parent samples of size $n_{\rm P, 2}$. The term ``$+1$'' reduces the bias especially when $\widetilde{H}_{\rm PO, 2}$ is small \cite[e.g.,][]{ecolevol2021p}, where a similar derivation of this bias correction is provided in \cite{Akita_2019}. By replacing $N_2$ by $\widehat{N_2}$ in Eqs.~\ref{AP-1} and \ref{AP-2}, we can obtain the following estimators: 

\begin{align}
\widehat{M_4} &=  \frac{ N_{1} \left(\widetilde{H}_{\rm HS,bet} + \widetilde{H}_{\rm PO,bet}\right) \left(2 n_{\rm O, 2} n_{\rm P, 2} + 1\right)} {2n_{\rm O, 1} \left(2 n_{\rm O, 2} + n_{\rm P, 2}\right) \left(\widetilde{H}_{\rm PO, 2} + 1\right)},
\label{AP-4}
\end{align}
and
\begin{align}
\widehat{m_4} &=  \frac{ \left(\widetilde{H}_{\rm HS,bet} + \widetilde{H}_{\rm PO,bet}\right) \left(2 n_{\rm O, 2} n_{\rm P, 2} + 1\right)} {2n_{\rm O, 1} \left(2 n_{\rm O, 2} + n_{\rm P, 2}\right) \left(\widetilde{H}_{\rm PO, 2} + 1\right)}.
\label{AP-5}
\end{align}
While estimator of the migrate number (Equation~\ref{AP-4}) requires the (unknown) parent number in the population 1 ($N_{1}$), estimator of the migration rate (Equation~\ref{AP-5}) can be obtained only the observed kinship pairs. 

Similar to estimating $N_2$, when PO pairs found in offspring samples and parent samples both from the population 1 in the first year are also available, we can obtain the estimator of $N_1$, written by
\begin{align}
\widehat{N_1} &= \frac{ 2 n_{\rm O, 1} n_{\rm P, 1} + 1 }{\widetilde{H}_{\rm PO, 1} + 1},
\label{AP-6}
\end{align}
where $\widetilde{H}_{\rm PO, 1}$ is the observed number of PO pairs found in offspring samples of size $n_{\rm O, 1}$ and parent samples of size $n_{\rm P, 1}$. By replacing $N_1$ by $\widehat{N_1}$ in Eq.~\ref{AP-4}, we can obtain the estimator of migration number that can be calculated only from genetic data, given by
\begin{align}
\widehat{M_5} &=  \frac{ \left(\widetilde{H}_{\rm HS,bet} + \widetilde{H}_{\rm PO,bet}\right) \left(2 n_{\rm O, 1} n_{\rm P, 1} + 1\right) \left(2 n_{\rm O, 2} n_{\rm P, 2} + 1\right)} {2n_{\rm O, 1} \left(2 n_{\rm O, 2} + n_{\rm P, 2}\right) \left(\widetilde{H}_{\rm PO, 1} + 1\right) \left(\widetilde{H}_{\rm PO, 2} + 1\right)}.
\label{AP-7}
\end{align}

Table \ref{parameter} summarizes the conditions for estimators appeared in this paper as to whether $N_1$ and/or $N_2$ are known and which type of samples are needed for estimation, and provides required kinship type for calculating the estimator. 

\subsection{Variation in reproductive potential among individuals and populations}

The proposed estimators are derived under the assumption that reproductive potential is the same among individuals. As noted in {\bf Appendix 2}, this assumption can be relaxed and the estimators still hold, which is exemplified in iteroparous species that may show significant variation in reproductive potential among individuals. Furthermore, the estimators even hold for most of the case where mean reproductive potential are different between population 1 and 2, which is exemplified in the situation that the environmental condition for reproductive success varies between populations. The necessary condition is that the migration event does not depends on the degree of reproductive potential (see details in  {\bf Appendix 3}). 

\subsection{Individual-based model}

We developed an individual-based model that tracks kinship relationships to evaluate the estimator's performance. The population structure was assumed to be identical to that in the development of the estimators. The population 1 and 2 comprised $N_1$ and $N_2$ parents with an equal sex ratio, and their offspring number was assumed to follow the geometric distribution with mean ${\bar \lambda_1}$ and ${\bar \lambda_2}$ (i.e., Poisson reproduction with mean $\lambda_{i,1}$ and $\lambda_{j,2}$ which follows the exponential distribution with mean ${\bar \lambda_1}$ and ${\bar \lambda_2}$), respectively. Migrant parents were randomly chosen from the population 1 at the end of the first year. Each offspring retained the parent's ID, making it possible to trace an HS and PO relationship.

Let a parameter set ($N_1$, $N_2$, $M$, ${\bar \lambda_1}$, ${\bar \lambda_2}$, $n_{\rm O,1}$, $n_{\rm O,2}$, $n_{\rm P,1}$, $n_{\rm P,2}$) be given. We simulated a population history and a sampling process, which generates proposed estimators; this process was repeated 1000 times, making it possible to construct the distribution of the estimators for each parameter set. All scripts and documentation for these analyses are available at \\https://github.com/teTUNAakita/CKMRmig.

\section{Results}\label{sec3}

We numerically evaluated the performance of ${\hat M}$s or ${\hat m}$s for the case in which variable reproductive potential among parents. Scaled statistical properties of ${\hat m_1}$,  ${\hat m_2}$,  ${\hat m_3}$ and ${\hat m_4}$ are completely the same to ${\hat M_1}$, ${\hat M_2}$, ${\hat M_3}$ and ${\hat M_4}$, hereafter, we only demonstrate the results of ${\hat M_1}$-${\hat M_5}$. {\bf Figure~\ref{violins}} illustrates the distribution of the relative bias of ${\hat M}$ for limiting cases where parent and offspring sample numbers are identical (i.e., $n=n_{\rm P,1}=n_{\rm P,2}=n_{\rm O,1}= n_{\rm O,2}$) and parent numbers in the two populations are also identical (i.e., $N=N_1=N_2$). Relative bias is calculated by applying outputs of the individual-based model, which is defined as follows: ``averaged estimator - true value/true value.'' Full list of parameter sets to evaluate the performance (relative bias and CV of ${\hat M}$s) is summarized in {\bf Table S1} in {\bf Supporting Information}. 
 
%%%%%%%%%%%%%%
\begin{center}
\fbox{Figure \ref{violins} is here. }
\end{center}
%%%%%%%%%%%%%%

First, we evaluated the accuracy of ${\hat M}$s based on their relative bias. As expected, for most of the investigated parameter sets, we observed that their relative bias is less than 5\%, as demonstrated in {\bf Table S1} in {\bf Supporting Information}. Thus, when the assumption that kinships are detected without any error is satisfied, it is reasonable to call ${\hat M}$s (nearly) unbiased estimators. 

Next, we evaluated the precision of ${\hat M}$s based on their CV. {\bf Table S1} in {\bf Supporting Information} demonstrated the value of the CV; meanwhile, the violin plot ({\bf Figure~\ref{violins}}) visually provides the degree of precision. For each estimator ${\hat M}$, the precision increases with increasing of sample size. It should be noted that total sample size depends on the estimator. For example, $n=50$ in Fig.~\ref{violins}, the total sample size of ${\hat M_1}$, ${\hat M_2}$, ${\hat M_3}$, ${\hat M_4}$ and ${\hat M_5}$ is 100 ($=n_{\rm O,1}+n_{\rm O,2}$), 100 ($=n_{\rm O,1}+n_{\rm P,2}$), 150 ($=n_{\rm O,1}+n_{\rm O,2}+n_{\rm P,2}$), 150 ($=n_{\rm O,1}+n_{\rm O,2}+n_{\rm P,2}$) and 200 ($=n_{\rm O,1}+n_{\rm O,2}+n_{\rm P,1}+n_{\rm P,2}$), respectively. 

As sample size or migrants increase, the precision increases and the shape of the distribution asymptotically becomes symmetric ({\bf Figure~\ref{violins}}). This is because an increasing of sample size or migrants is likely to increase the observed number of kinship pairs found between two populations ($\widetilde{H}_{\rm HS,bet}$ or $\widetilde{H}_{\rm PO,bet}$) and decreases the variance of those kinship pair numbers. In addition, the number of PO pair within a population ($\widetilde{H}_{\rm PO,1}$ or $\widetilde{H}_{\rm PO,2}$) also contributes the precision for estimating $N_1$ or $N_1$, respectively, providing ${\hat M_4}$ and ${\hat M_5}$ with a higher precision. 

Our simulator can handle the situation with or without invasive sampling. Invasive sampling potentially affects the degree of $m$ since sampled parents in the population 1 have no chance to move to the population 2 and thus those parents cannot be sampled. In this case, $m$ should be defined by $M/(N_1-n_{P,1})$ not by $M/N_1$, although it does not affect the estimator ${\hat M}$s, as demonstrated in {\bf Table S1} ({\bf Supporting Information}). 

Finally, we investigated the situation in which the parental number in the population 1 is much more than in the population 2 (i.e., $N_1 \gg N_2$) and thus migrants are large proportion of $N_2$. In such a setting, e.g., $(N_1,N_2,M)=(10^4,10^3,500)$, we confirmed the same property of ${\hat M}$s to the situation $N_1 = N_2$, as demonstrated in {\bf Table S1} ({\bf Supporting Information}), suggesting the robustness of ${\hat M}$s in heterogeneous population sizes.   

\section{Discussion}\label{sec4}

We theoretically developed unbiased estimators of the contemporary migration number (${\hat M_1}$--${\hat M_5}$) and migration rate (${\hat m_1}$--${\hat m_4}$) of parents between two populations in iteroparous species, where the migration direction is specified. The proposed estimators are based on known PO relationship and HS relationships observed between and within two populations without any error of kinship assignment. Users can choose the estimator developed for the situation as to whether the parental number of population 1 ($N_1$) and/or that of population 2 ($N_2$) is known ({\bf Table~\ref{parameter}}). The performance of the estimator (accuracy and precision) was quantitatively evaluated by running an individual-based model ({\bf Figure~\ref{violins}}; see also {\bf Table S1} in {\bf Supporting Information}). Our modeling framework utilizes several types of reproductive variations (survived offspring number per parent), including variance of reproductive potential between and within populations, taking account of several situations, including a body-size structure or environmental heterogeneity for reproductive success.

We presented mainly three novel points in this paper. First, we formulated probabilities of kinship pairs randomly chosen between two populations ($\pi_{\rm HS,bet}$ and $\pi_{\rm PO,bet}$), which provide estimators of the migration number or rate. While similar derivations are found in (nongenetical) mark-recapture method (\RED{REF}), these are limited to the situation such that sampling can be non-invasive and adult individuals are sampled. Our proposed method can avoid these limitations, where such an advantages is characterized by CKMR method. Second, we revealed that the probabilities of kinship pairs are approximately independent of reproductive potential and thus information about it is not needed for the estimation, if migration event and the potential are independent. This is a useful property of the estimators because heterogeneity of reproductive potential within/between populations are generally observed in iteroprous species. Third, we demonstrated applicability that the estimation can be obtained only from the genetic data (i.e., ${\hat M_5}$ and ${\hat m_4}$). There are several advantages in using the proposed estimator instead of separately estimating $\pi_{\rm HS,bet}$ (and/or $\pi_{\rm PO,bet}$) (via genetic method) and $N_1$ (and/or $N_2$) (via nongenetical method), including simplified sampling and analyzing designs and availability of unified framework of genetic analyzes for detecting PO and HS pairs (similar discussion is found for estimating effective breeding size per census size, $N_{\rm b}/N$, in \cite{Akita_2020}). 

To estimate the contemporary migration number or migration rate, our simulation-based results provide guidance for a sample size to ensure the required accuracy and precision, especially if the order of the number of migrant parents and parental sizes are approximately known ({\bf Table S1} in {\bf Supporting Information}). For example, when $m=0.1$ and $N_1=N_2=10^3$, sampling 10\% and 20\% parents and an equal number of offspring in two populations leads 70\% and 39\% CV of ${\hat M_5}$, respectively (case of invasive sampling). Even if there is no information about these numbers, $1/\sqrt{H_{\rm HS, bet}}$ (or $1/\sqrt{H_{\rm PO, bet}}$) provides an approximate lower bound on the CV, which can be used as an indicator of the precision of ${\hat M}$ or ${\hat m}$. Furthermore, the condition that $n>\sqrt{N_2}$ is also used as rule of thumb especially when planning a research project (see also Equations~\ref{RS-1} and \ref{RS-2}). 

There may be several cases where it is desirable to apply the proposed estimator. The first case is where there is a certain number of migrants between populations, e.g., $M>100$, which is too large to be detected by population genetics method if $M$ is interpreted as effective migration number. Information on movement between populations is essential for assessing population dynamics in the context of conservation and management, even if migrant number is so large that it cannot be genetically assigned to two populations. The second is the case of the application to genetic monitoring conducted every reproductive timing (e.g., annually). This is because proposed estimators have information about contemporary migration, thus the time series data for migration reflect environmental changes, which provides insights into the underlying ecological processes. In addition, HS pairs found within the genetic population provides $N_{\rm b}$ \cite[]{wang2009new}, which is also used for assessing a genetic health. The third case is where sampling adults is difficult due to conservation purpose or other reasons instead sampling offspring is relatively easy. In this case, although $N_2$ (and $N_1$) need to be given externally for the usage of $\hat m_1$ (and $\hat M_1$), there is presently no other method for estimating adult movement especially when two populations cannot be genetically distinct, indicating that proposed method has the potential to greatly widen the scope of population monitoring. 

Finally, we note some caveats in applying proposed method. Our theory for developing estimators assumes that kinships are detected without any error. There are many algorithms to detect kinship pairs from single nucleotide polymorphisms (SNPs) or short tandem repeats (STRs) \cite[e.g., ][]{Wang_2009Genetics, Huisman_2017}, although a HS pair requires many more DNA markers than a PO pair. In addition, iteroprous species potentially has several kinship types such as halfsib-uncle/nephew or halfsib-cousin, which are expected to appear frequently and should be correctly distinguished from HS pairs. It is desirable to estimate in advance how many markers are needed for kinship detections in target populations associated with simulation of a pedigree reconstruction including distant kinships (e.g., \cite{Anderson_2020}). Our proposed estimators are limited to detect one-time parental movements on a period between breeding seasons in given populations. Thus, the estimation of fine-scale spatial-temporal movements, which is available to integrate multiple type of data \cite[]{Thorson_2021}, goes beyond the scope of this paper, although population dynamics models with a coarser spatial resolution than the spatial scale for environmental layers are often used in the assessment models, which is the target of application of the proposed method. In addition, the current theory behind the estimators does not assume desynchronized reproduction within a population (e.g., skip spawning) and correlation between mobility and fertility. Our proposed estimators require the pre-specification of a population structure. Kinship relationships with information of a sample location potentially provide the chance to explore a plausible population structure via estimating migration number or rate between hypothetical populations, which contributes the determination of a management unit \cite[]{Hastings_1993, Waples_2006_MolEcol, PALSBOLL200711}. This perspective of the study will be conducted in the future.

%  (e.g., \cite{Hastings_1993} numerically demonstrates that two populations become demographically independent when migration rate $< 0.1$)
% withinのデータはどう使う??withinとbetweenを比較して、betweenが優位に少ないかどうかなどが構造発見につながるかもconn et al 2020 EcolEvolもチェック
%% 今回提案した方法は、ゴーストポップがあっても大丈夫

%%%%%%%%%%%%%%%%%%%%%%%%%%%%%%%%%
\section*{ACKNOWLEDGMENTS}
The author thanks Y. Tsukahara and N. Suzuki for fruitful discussions. This work was supported by JSPS KAKENHI Grant Number 19K06862 and 20H03012.

\section*{CONFLICT OF INTEREST}
The author declares no conflict of interest.

\section*{AUTHOR CONTRIBUTIONS}
{\bf Tetsuya Akita:} Conceptualization (lead); formal analysis (lead); funding acquisition (lead); methodology (lead); writing--original draft (lead); writing--review and editing (lead).
%%%%%%%%%%%%%%%%%%%%%%%%%%%%%%%%%

\section*{DATA ACCESSIBILITY}
No datasets were generated or analyzed in this study.

\bibliographystyle{apacite}
\bibliography{akita}

\clearpage

\renewcommand{\arraystretch}{0.6}
\begin{table}[tb]
%\centering
   \caption[]{Table 1: The list of mathematical symbols employed in the main text}
    \textbf {}\\[-4mm]
    \begin{tabular}{llc} \hline
       & & \\
	$n_{\rm P,1}, n_{\rm P, 2}$			& Sampled number of parents from the population 1 and 2\\ 
		                						& \\
	$n_{\rm O, 1}, n_{\rm O, 2}$			& Sampled number of offspring from the population 1 and 2\\ 
		                						& \\
	$N_{1}, N_{2}$						& Number of parents in the population 1 and 2 when sampled offspring are born\\
		                						& \\
	$M$								& Number of survived migrants of parents from the population 1 to population 2\\
		                						& \\
	$m$								& Migration rate of parents from the population 1 to population 2, defined by $M/N_1$.\\
		                						& \\
	$r$								& Sex ratio\\
		                						& \\
	$\pi_{\rm PO,1}, \pi_{\rm PO,2}$		& Probability that a randomly selected pair (parent and offspring) \\
	                							& shares a parent-offspring relationship within the population 1 and 2\\
									& \\
	$\pi_{\rm PO, bet}$					& Probability that a randomly selected pair (parent and offspring) \\
	                							& shares a parent-offspring relationship between the population 1 and 2\\
									& \\
	$\pi_{\rm HS, bet}$					& Probability that a randomly selected pair (two offspring) \\
	                							& shares a half-sibling relationship between the population 1 and 2\\
					                			& \\
	$\lambda_{i,1}, \lambda_{j,2}$			& Expected number of surviving offspring of parent $i$ and $j$ at sampling in the population 1 and 2\\
		                						& \\
	$\lambda_{M,l,1}, \lambda_{M,l,2}$		& Expected number of surviving offspring of migrant $l$ at sampling in the population 1 and 2\\
		                						& \\
	$k_{i,1}, k_{j,2}$					& Number of surviving offspring born to parent $i$ and $j$ in the population 1 and 2\\ 
	                							& \\
	$H_{\rm PO, 1}, H_{\rm PO, 2}$		& Number of parent-offspring pairs observed in samples within the population 1 and 2\\ 
	                							& \\
	$H_{\rm PO, bet}$					& Number of parent-offspring pairs observed in samples between the population 1 and 2\\ 
	                							& \\
	$H_{\rm HS, bet}$					& Number of half-sibling pairs observed in samples between the population 1 and 2\\ 
	                							& \\              		
	\hline
    \end{tabular}
    \label{symbols} 
\\Subscription ``1'' and ``2'' indicates the quantity in the population 1 during the first year and in the population 2 during the second year, respectively. 
\end{table}
\renewcommand{\arraystretch}{1}

\clearpage

\renewcommand{\arraystretch}{0.6}
\begin{table}[tb]
\begin{center}
   \caption[]{Table 2: Summary of proposed estimators for required parameters and kinship types}
    \textbf {}\\[-0mm]
    \begin{tabular}{ccccccccc} \hline
       	\\
	Estimator			& $N_1$ 		& $N_2$ 		& $n_{\rm O, 1}$	& $n_{\rm O, 2}$	& $n_{\rm P, 1}$	& $n_{\rm P, 2}$	& Required kinship type\\
	\\
	\hline
	\\
	$\widehat{M_1}$	& given 		& given 		& \checkmark		& \checkmark 		&				&				& HSP\\ 
		                						& \\
	$\widehat{M_2}$	& given		& given		& \checkmark		& 				&				& \checkmark 		& POP\\
		                						& \\
	$\widehat{M_3}$	& given		& given		& \checkmark		&  \checkmark		&				& \checkmark 		& HSP \& POP\\
		                						& \\
	$\widehat{M_4}$	& given		& estimated	& \checkmark		&  \checkmark		&				& \checkmark 		& HSP \& POP\\
		                						& \\
	$\widehat{M_5}$	& estimated	& estimated	& \checkmark		&  \checkmark		& \checkmark		& \checkmark 		& HSP \& POP\\
									& \\
	$\widehat{m_1}$	& ---			& given		& \checkmark		& \checkmark 		&				&				& HSP\\ 
									& \\
	$\widehat{m_2}$	& ---			& given		& \checkmark		& 				&				& \checkmark 		& POP\\					                			& \\
	$\widehat{m_3}$	& ---			& given 		& \checkmark		&  \checkmark		&				& \checkmark 		& HSP \& POP\\
		                						& \\
	$\widehat{m_4}$	& ---			& estimated	& \checkmark		&  \checkmark		& 				& \checkmark 		& HSP \& POP\\
		                						& \\
	\hline
    \end{tabular}
    \label{parameter} 
\end{center} 
\end{table}
\renewcommand{\arraystretch}{1}

\clearpage

\newcommand{\figcapa}{(a) Hypothetical populations with $N_{1}=16$, $N_{2}=14$, and $M=6$. Upper and lower square indicates individuals belonging to the population 1 before migration (at sample timing in the first year) and individuals belonging to the population 2 after reproduction (at sample timing in the second year), respectively. The open circles on the left side, right side, and the center represent mothers, fathers, and their offspring, respectively. The thin line denotes parent-offspring relationship. The bold arrow denotes the migration and the symbol x denotes a failure to survive at sampling in the second year. Sampled individuals are labeled with an index number. The number of sampled individuals in this example: $n_{\rm P, 1}=4$, $n_{\rm O, 1}=5$, $n_{\rm P, 2}=6$, $n_{\rm O, 2}=4$; the numbers of kinship pairs: $H_{\rm PO, bet}=1$ (i.e., ``7-10'' pair), $H_{\rm HS, bet}=2$ (i.e., ``8-16'' and ``9-16'' pairs), $H_{\rm PO, 1}=3$ (i.e., ``1-6'', ``3-5'' and ``4-7'' pairs), and $H_{\rm PO, 2}=2$ (i.e., ``11-17'' and ``15-16'' pairs). (b) The phases of the events relevant to this study along the time line.}

\newcommand{\figcapb}{Violin plots showing the distribution of relative bias in our estimator of $M$ for various values sample sizes, parent numbers and migration rate. Filled circles represent the mean values. Sample sizes for parents and offspring are identical (i.e., $n=n_{\rm P,1}=n_{\rm P,2}=n_{\rm O,1}= n_{\rm O,2}$) and parent sizes in the two populations are also identical (i.e., $N=N_1=N_2$), which are indicated in the legend. Migration rate $m$ is specified by $N_1/M$. For the demonstration purposes, the upper side of the distribution is truncated, although the mean values are calculated including the truncated values. }

%%%FIGURE 1%%%
\begin{figure}[!h]
	\begin{center}
		\includegraphics[width=0.7\textwidth]{/Users/akita/Dropbox/research/close_kin/Mig/Figs/cartoon.eps}
		\caption{{\bf FIGURE 1} \figcapa{}}
		\label{cartoon}
	\end{center}
\end{figure}
%%%%%%%%%%%

%%%FIGURE 2%%%
\begin{figure}[!h]
	\begin{center}
		\includegraphics[width=1\textwidth]{/Users/akita/Dropbox/research/close_kin/Mig/Figs/violins.eps}
		\caption{{\bf FIGURE 2} \figcapb{}}
		\label{violins}
	\end{center}
\end{figure}
%%%%%%%%%%%



\clearpage

\section*{APPENDIX 1}
\setcounter{equation}{0}
\renewcommand{\theequation}{A\arabic{equation}}

\section*{Reproductive potential}

\renewcommand{\theequation}{A\arabic{equation}}
Here, we introduce the concept of the reproductive potential of parent $i$ and $j$ in the population 1 and 2, respectively, which are defined as the expected number of surviving offspring at sampling time, denoted by $\lambda_{i,1}$ and $\lambda_{j,2}$ ($i=(1, \ldots, N_1)$ and $j=(1, \ldots, N_2)$). The reproductive potential is determined by several factors, including the parent's age, weight, residence time on the spawning ground. It should be noted that the magnitude of this parameter includes information about the survival rate of the offspring, the number of days after egg hatching, and the egg number; this implies that the parameter reflects the sample timing. It should also be noted that the modeling framework does not depend on whether the reproductive potential is heritable or not.

\section*{APPENDIX 2}
\section*{Derivation of Equations~\ref{HS-1} and \ref{PO-1} when reproductive potential is variable among parents}

In the main text, we ignore the variation of reproductive potential among parents (i.e., both $\lambda_{i,1}$ and $\lambda_{j,2}$ are constant) to derive $\pi_{\rm HS,bet}$ and $\pi_{\rm PO,bet}$. Let $k_{i,1}$ and $k_{j,2}$ be the number of surviving offspring of parent $i$ and $j$ at sampling in the population 1 and 2, respectively, and $k_{i,1}$ and $k_{j,2}$ is assumed to follow a kind of discrete distribution (e.g., negative binomial distribution) with mean $\lambda_{i,1}$ and $\lambda_{j,2}$. Without loss of generality, we set the index such that parents with $i=1$ to $M$ in the population 1 and $j=1$ to $M$ in the population 2 are identical migrants; for example, the parent with $i=1$ reproduce $k_{1,1}$ offspring (in the population 1) and then reproduce $k_{1,2}$ offspring (in the population 2) after migration. Giving $k_{i,1}$ and $k_{j,2}$, the conditional probability that two offspring sampled in the population 1 and 2 share an HS relationship is
\begin{align}
\pi_{\rm HS,bet} | _{\boldsymbol{k_1, k_2}} &= \frac{ 2\sum_{i=1}^M k_{i,1} } { \sum_{i=1}^{N_1} k_{i,1} } \frac{ 2\sum_{j=1}^M k_{j,2} } { \sum_{j=1}^{N_2} k_{j,2} } \frac{1}{M}, 
\end{align}
where $\boldsymbol{k_1}=(k_{1,1}, \ldots, k_{M,1}, \ldots, k_{N_1,1})$ and $\boldsymbol{k_2}=(k_{1,2}, \ldots, k_{M,2}, \ldots, k_{N_2,2})$. It should be noted that $k_{i,1}$ and $k_{j,1}$ is a random variable with mean $\lambda_{i,1}$ and $\lambda_{j,2}$, respectively. By taking the expectation over the distribution of offspring number, the conditional probability is approximately given by
\begin{align}
\pi_{\rm HS,bet} | _{\boldsymbol{\lambda_1, \lambda_2}} &= \mathbb{E}[\pi | _{\boldsymbol{k_1, k_2}}] \nonumber\\
&= \frac{4}{M} \mathbb{E}\left[\frac{ \sum_{i=1}^M k_{i,1} } { \sum_{i=1}^{N_1} k_{i,1} } \frac{ \sum_{j=1}^M k_{j,2} } { \sum_{j=1}^{N_2} k_{j,2} } \right]   \nonumber\\
&\approx \frac{4}{M} \frac{ \mathbb{E}\left[ \sum_{i=1}^M k_{i,1} \sum_{j=1}^M k_{j,2} \right] } {\mathbb{E}\left[ \sum_{i=1}^{N_1} k_{i,1} \sum_{j=1}^{N_2} k_{j,2}\right]} \nonumber\\
&= \frac{4}{M} \frac{ \sum_{i=1}^M \lambda_{i,1} \sum_{j=1}^M \lambda_{j,2} } {\sum_{i=1}^{N_1} \lambda_{i,1} \sum_{j=1}^{N_2} \lambda_{j,2}},
\end{align}
where $\boldsymbol{\lambda_1}=(\lambda_{1,1}, \ldots, \lambda_{M,1}, \ldots, \lambda_{N_1,1})$ and $\boldsymbol{\lambda_2}=(\lambda_{1,2}, \ldots, \lambda_{M,2}, \ldots, \lambda_{N_2,2})$. From the second to the third line, we use the approximation that $\mathbb{E}[g1(k)/g2(k)] \approx \mathbb{E}[g_1(k)] / \mathbb{E}[g_2(k)]$. From the third to the forth line, we use the relationship that $\mathbb{E}[k_{i,1}k_{j,2} | _{\lambda_{i,1}, \lambda_{j,2}}] = \mathbb{E}[k_{i,1}|_{\lambda_{i,1}}]\mathbb{E}[k_{j,2}|_{\lambda_{j,2}}]$, implying statistical independence of offspring number before and after parental movement. In other words, the conditional probability does not affected by $\mathbb{V}[k|_\lambda]$. We assume that $\lambda_{i,1}$ and $\lambda_{j,2}$ is also a random variable followed by an arbitral function with mean ${\bar \lambda_{1}}$ and ${\bar \lambda_{2}}$, respectively. By taking the expectation over $\lambda$ and applying a similar approximation, the unconditional probability is given by
\begin{align}
\pi_{\rm HS,bet} &= \mathbb{E}[\pi | _{\boldsymbol{\lambda_1, \lambda_2}}] \nonumber\\ 
&= \frac{4}{M} \mathbb{E}\left[\frac{ \sum_{i=1}^M \lambda_{i,1} } { \sum_{i=1}^{N_1} \lambda_{i,1} } \frac{ \sum_{j=1}^M \lambda_{j,2} } { \sum_{j=1}^{N_2} \lambda_{j,2} } \right]   \nonumber\\
&\approx \frac{4}{M} \frac{ \mathbb{E}\left[ \sum_{i=1}^M \lambda_{i,1} \sum_{j=1}^M \lambda_{j,2} \right] } {\mathbb{E}\left[ \sum_{i=1}^{N_1} \lambda_{i,1} \sum_{j=1}^{N_2} \lambda_{j,2}\right]} \nonumber\\
&= \frac{4}{M} \frac{M^2 {\bar \lambda_1} {\bar \lambda_2}}{N_1 N_2 {\bar \lambda_1} {\bar \lambda_2}}\nonumber\\
&= \frac{4M}{N_{1}N_{2}},
\label{App2}
\end{align}
which provides the same formulation described in Eq.\ref{HS-1}. It should be noted that, from the third to the forth line, we assume that $\lambda_{l,1}$ and $\lambda_{l,2}$ ($l = 1, \ldots, M$) are independent variables (i.e., $\mathbb{E}[\lambda_{l,1}\lambda_{l,2}] = {\bar \lambda_{1}}{\bar \lambda_{2}}$), implying variable reproductive potential of an identical parent before and after migration. 

Next, we derive the probability that offspring sampled in the population 1 and a parent sampled in the population 2 share an PO relationship ($\pi_{\rm PO,bet}$) under the flexible settings of $\lambda$, which is similar to the derivation of $\pi_{\rm HS,bet}$. The conditional probability is given by
\begin{align}
\pi_{\rm PO,bet} | _{\boldsymbol{k_1, k_2}} &= \frac{ 2\sum_{i=1}^M k_{i,1} } { \sum_{i=1}^{N_1} k_{i,1} } \frac{ \sum_{j=1}^M k_{j,2} } { \sum_{j=1}^{N_2} k_{j,2} } \frac{1}{M}.
\end{align}
By taking the expectation over $k$ and $\lambda$, in the same manner as noted above, the unconditional probability is approximately given by
\begin{align}
\pi_{\rm PO,bet} &= \mathbb{E}[\mathbb{E}[\pi | _{\boldsymbol{k_1, k_2}}]] \nonumber\\ 
&\approx  \frac{2M}{N_{1}N_{2}},
\end{align}
which provides the same formulation described in Eq.\ref{PO-1}.

\section*{APPENDIX 3}

\section*{Difference in reproductive potential between migrants and non-migrants}

In the process of the derivation of $\pi_{\rm HS,bet}$ and $\pi_{\rm PO,bet}$ in {\bf APPENDIX 2}, we ignore covariation between migration and reproductive potential. Here, we consider the situation in which migrants have a distinguishable distribution of reproductive potential from non-migrant parents. Let $\lambda_{i,1}$, $\lambda_{j,2}$, $\lambda_{M,l,1}$ and $\lambda_{M,l,2}$ be a reproductive potential of non-migrant in the population 1 and 2 and migrants in the population 1 and 2 with mean ${\bar \lambda_{1}}$, ${\bar \lambda_{2}}$, ${\bar \lambda_{M,1}}$ and ${\bar \lambda_{M,2}}$, respectively. Under this setting, $\pi_{\rm HS,bet}$ can be written by
\begin{align}
\pi_{\rm HS,bet} &\approx \frac{4}{M}  \frac{ \mathbb{E}\left[ \sum_{l=1}^M \lambda_{M,l,1} \sum_{l=1}^M \lambda_{M,l,2} \right]} {\mathbb{E}\left[ \left(\sum_{l=1}^{M} \lambda_{M,l,1} + \sum_{i=M+1}^{N_1} \lambda_{i,1}\right) \left(\sum_{l=1}^{M} \lambda_{M,l,2} + \sum_{j=M+1}^{N_2} \lambda_{j,2}\right) \right] }\nonumber\\ 
&= \frac{4}{M} \frac{ M^2 {\bar \lambda_{M,1}} {\bar \lambda_{M,2}} } { M {\bar \lambda_{M,1}} + (N_1 - M) {\bar \lambda_1} + M {\bar \lambda_{M,2}} + (N_2 - M) {\bar \lambda_2} }.
\label{App3}
\end{align}
If ${\bar \lambda_{M,1}} = {\bar \lambda_{1}}$ and ${\bar \lambda_{M,2}} = {\bar \lambda_{2}}$, Eq.~\ref{App3} reduces to Eq.\ref{App2}. This formulation include reproductive potential terms, so that distinct reproductive potential between migrants and non-migrants vanishes the usefulness from proposed HS-based estimators (this is also applied to PO-based estimators proposed in this paper). 

\clearpage

%\begin{comment}
%%%%%%%%%%%%%%%%%%%%%%%%%%%%%%%%%
%\section*{Supporting Information}

%\setcounter{figure}{0}
%\newcommand{\figcapsa}{Heatmap showing the relative bias of $\widehat{N_{\rm e,m}/N_{\rm m}}$ as a function of both $n_{\rm M}$ and $n_{\rm O}$. The value of the relative bias is indicated in the legend. The value of the combined effect of parental and nonparental variations increases from left to right ($c=1, 10, 20$, and $100$). (a) $N_{\rm m}=1{,}000$, (b) $N_{\rm m}=10{,}000$. }

%\newcommand{\figcapsb}{Heatmap showing the relative bias of $\widehat{N_{\rm e,m}}$ as a function of both $n_{\rm M}$ and $n_{\rm O}$. The value of the relative bias is indicated in the legend. The value of the combined effect of parental and nonparental variations increases from left to right ($c=1, 10, 20$, and $100$). (a) $N_{\rm m}=1{,}000$, (b) $N_{\rm m}=10{,}000$. }

%\newcommand{\figcapsc}{Heatmap showing the relative bias of $\widehat{1/N_{\rm m}}$ as a function of both $n_{\rm M}$ and $n_{\rm O}$. The value of the relative bias is indicated in the legend. The value of the combined effect of parental and nonparental variations increases from left to right ($c=1, 10, 20$, and $100$). (a) $N_{\rm m}=1{,}000$, (b) $N_{\rm m}=10{,}000$. }

%\newcommand{\figcapsd}{Heatmap showing the coefficient of variation of $\widehat{N_{\rm e,m}/N_{\rm m}}$ as a function of both $n_{\rm M}$ and $n_{\rm O}$. The value of the coefficient of variation is indicated in the legend. The value of the combined effect of parental and nonparental variations increases from left to right ($c=1, 10, 20$, and $100$). (a) $N_{\rm m}=1{,}000$, (b) $N_{\rm m}=10{,}000$.}

%\newcommand{\figcapse}{Heatmap showing the coefficient of variation of $\widehat{N_{\rm e,m}}$ as a function of both $n_{\rm M}$ and $n_{\rm O}$. The value of the coefficient of variation is indicated in the legend. The value of the combined effect of parental and nonparental variations increases from left to right ($c=1, 10, 20$, and $100$). (a) $N_{\rm m}=1{,}000$, (b) $N_{\rm m}=10{,}000$.}

%\newcommand{\figcapsf}{Heatmap showing the coefficient of variation of $\widehat{1/N_{\rm m}}$ as a function of both $n_{\rm M}$ and $n_{\rm O}$. The value of the coefficient of variation is indicated in the legend. The value of the combined effect of parental and nonparental variations increases from left to right ($c=1, 10, 20$, and $100$). (a) $N_{\rm m}=1{,}000$, (b) $N_{\rm m}=10{,}000$.}

%%%FIGURE S1%%%
%\begin{figure}[!h]
%	\begin{center}
%		\includegraphics[width=1\textwidth]{figs/NeN_bias.pdf}
%		\caption{{\bf FIGURE S1} \figcapsa{}}
%		\label{NeN_bias}
%	\end{center}
%\end{figure}
%%%%%%%%%%%

%%%FIGURE S2%%%
%\begin{figure}[!h]
%	\begin{center}
%		\includegraphics[width=1\textwidth]{figs/Ne_bias.pdf}
%		\caption{{\bf FIGURE S2} \figcapsb{}}
%		\label{Ne_bias}
%	\end{center}
%\end{figure}
%%%%%%%%%%%

%%%FIGURE S3%%%
%\begin{figure}[!h]
%	\begin{center}
%		\includegraphics[width=1\textwidth]{figs/Ninv_bias.pdf}
%		\caption{{\bf FIGURE S3} \figcapsc{}}
%		\label{Ninv_bias}
%	\end{center}
%\end{figure}
%%%%%%%%%%%

%%%FIGURE S4%%%
%\begin{figure}[!h]
%	\begin{center}
%		\includegraphics[width=1\textwidth]{figs/NeN_cv.pdf}
%		\caption{{\bf FIGURE S4} \figcapsd{}}
%		\label{NeN_cv}
%	\end{center}
%\end{figure}
%%%%%%%%%%%

%%%FIGURE S5%%%
%\begin{figure}[!h]
%	\begin{center}
%		\includegraphics[width=1\textwidth]{figs/Ne_cv.pdf}
%		\caption{{\bf FIGURE S5} \figcapse{}}
%		\label{Ne_cv}
%	\end{center}
%\end{figure}
%%%%%%%%%%%

%%%FIGURE S6%%%
%\begin{figure}[!h]
%	\begin{center}
%		\includegraphics[width=1\textwidth]{figs/Ninv_cv.pdf}
%		\caption{{\bf FIGURE S6} \figcapsf{}}
%		\label{Ninv_cv}
%	\end{center}
%\end{figure}
%%%%%%%%%%%
%\clearpage
%%%%%%%%%%%%%%%%%%%%%%%%%%%%%%%%%%%%%%%%%%%%%%%
%\end{comment}

\end{spacing}
\end{document}